\documentclass[nobib]{tufte-handout}

%\\geometry{showframe}% for debugging purposes -- displays the margins

\newcommand{\bra}[1]{\left(#1\right)}
\usepackage{amssymb}
\usepackage{hyperref}
\usepackage{pgfplots}
\usepackage[activate={true,nocompatibility},final,tracking=true,kerning=true,spacing=true,factor=1100,stretch=10,shrink=10]{microtype}
\usepackage{color}
\usepackage{steinmetz}
\usepackage{placeins}
% Fixes captions and images being cut off
\usepackage{marginfix}
\usepackage{array}
\usepackage{tikz}
\usepackage{amsmath,amsthm}
\usetikzlibrary{shapes}
\usetikzlibrary{positioning}
\usepackage{listings}
\usepackage{forest}
\usepackage{caption}
\DeclareCaptionFont{white}{\color{white}}
\DeclareCaptionFormat{listing}{\colorbox{gray}{\parbox{\textwidth}{#1#2#3}}}
\captionsetup[lstlisting]{format=listing,labelfont=white,textfont=white}

% Set up the images/graphics package
\usepackage{graphicx}
\setkeys{Gin}{width=\linewidth,totalheight=\textheight,keepaspectratio}
\graphicspath{{.}}

\title{Notes for ECE 39595 - Object-Oriented Programming with C++}
\author{Zeke Ulrich}
\date{\today}  % if the \date{} command is left out, the current date will be used

% The following package makes prettier tables.  We're all about the bling!
\usepackage{booktabs}

% The units package provides nice, non-stacked fractions and better spacing
% for units.
\usepackage{units}

% The fancyvrb package lets us customize the formatting of verbatim
% environments.  We use a slightly smaller font.
\usepackage{fancyvrb}
\fvset{fontsize=\normalsize}

% Small sections of multiple columns
\usepackage{multicol}

% For finite state machines 
\usetikzlibrary{automata} % Import library for drawing automata
\usetikzlibrary{positioning} % ...positioning nodes
\usetikzlibrary{arrows} % ...customizing arrows
\tikzset{node distance=2.5cm, % Minimum distance between two nodes. Change if necessary.
    every state/.style={ % Sets the properties for each state
    semithick,
    fill=gray!10},
    initial text={}, % No label on start arrow
    double distance=2pt, % Adjust appearance of accept states
    every edge/.style={ % Sets the properties for each transition
    draw,
    ->,>=stealth', % Makes edges directed with bold arrowheads
    auto,
    semithick}}
\let\epsilon\varepsilon

% These commands are used to pretty-print LaTeX commands
\newcommand{\doccmd}[1]{\texttt{\textbackslash#1}}% command name -- adds backslash automatically
\newcommand{\docopt}[1]{\ensuremath{\langle}\textrm{\textit{#1}}\ensuremath{\rangle}}% optional command argument
\newcommand{\docarg}[1]{\textrm{\textit{#1}}}% (required) command argument
\newenvironment{docspec}{\begin{quote}\noindent}{\end{quote}}% command specification environment
\newcommand{\docenv}[1]{\textsf{#1}}% environment name
\newcommand{\docpkg}[1]{\texttt{#1}}% package name
\newcommand{\doccls}[1]{\texttt{#1}}% document class name
\newcommand{\docclsopt}[1]{\texttt{#1}}% document class option name

% Define a custom command for definitions and biconditional
\newcommand{\defn}[2]{\noindent\textbf{#1}:\ #2}
\let\biconditional\leftrightarrow

\begin{document}

\maketitle

\tableofcontents

\section{Course Description}
This course teaches C++ and the principles of object oriented programming.
It covers the basics of the C++ language, including inheritance, virtual
function calls and the mechanisms that support virtual function calls. Design
patterns and general principles of programming will be covered.
\pagebreak

\section{Introduction}
Let's examine some basic C++ programs.
\begin{lstlisting}[language=C++,caption=Hello World]
    #include <iostream>

    int main() {
        std::cout << "Hello World!";
        return 0;
    }
\end{lstlisting}

\begin{lstlisting}[language=C++,caption=User Input]
    #include <iostream>

    int main() {
        double n;
        int i;

        std::cout << "Enter float: ";
        std::cin >> n;

        std::cout << "Enter integer: ";
        std::cin >> i;

        return 0;
    }
\end{lstlisting}

The \texttt{stds} you're seeing all over the place refer not
to a frat party but the standard namespace. It holds
useful objects like standard in ("\texttt{cin}") and standard
out ("\texttt{cout}").

In C, we use \texttt{malloc} and \texttt{free} to allocate memory. In C++,
the equivalent operations are \texttt{new} and \texttt{delete}.

\begin{lstlisting}[language=C++,caption=Free and Delete]
    void CorrectUsage(){
        int *ptr = new int[3];
        int *ptr1 = new int;
        ptr[0] = 1;
        ptr[1] = 2;
        ptr[2] = 3;
        *ptr1 = 5;
        delete ptr1;
        delete [] ptr;        
\end{lstlisting}
The reason for this is using \texttt{new} and \texttt{free} calls
an object's constructor and destructors, which are defined to properly
delete the object. Technically, \texttt{malloc} and \texttt{free} are
both present in C++, but they won't trigger the constructors and
destructors and should be avoided. The choice to leave these functions in
was made to improve compatibility with C, which is a theme the reader
may notice in C++'s many possible ways to do the same thing.

C++ filenames are terminated with a \texttt{.cpp} or \texttt{.cc}
extension, like \texttt{example.cpp} or \texttt{example.cc}
\section{Objects and Classes}
A class stores functions (methods) and values for a bunch
of objects you may want to create that follow its blueprint,
like characters in a video game or car models.

\begin{lstlisting}[language=C++,caption=Objects and Classes]
    #include <iostream>

    class Dog {
            public:
            std::string breed;
            std::string name;

            void bark() {
                    std::cout << "Woof!" << std::endl;
                }
        };

    int main() {
            Dog my_dog;
            my_dog.breed = "Labrador";
            my_dog.name = "Buddy";

            Dog your_dog;
            your_dog.breed = "Poodle";
            your_dog.name = "Coco";

            std::cout << my_dog.breed << std::endl;
            my_dog.bark();

            return 0;
        }
\end{lstlisting}

There are many ways to create a new object in C++.

\begin{lstlisting}[language=C++,caption=Alternative Instantiations]
#include "Dog.h"
int main(int argc, char* argv[ ]) {
    Dog* myDog = new Dog("Sam");
    Dog myDog("Sanjay");
    Dog myDog = Dog("Jenna");
    ...
}
\end{lstlisting}

There are also many ways to access values in an object.
One potentially novel way is through the \texttt{->} operator.
It's shorthand for dereferencing a pointer and then accessing
a member of the object it points to.
\begin{lstlisting}[language=C++,caption=-> Operator]
class Example {
public:
    int value;
    void show() {
        std::cout << "Value: " << value << std::endl;
    }
};

int main() {
    Example obj;
    obj.value = 10;

    Example* ptr = &obj;  // Pointer to the object

    // Accessing members using the pointer
    ptr->value = 20;      // Equivalent to (*ptr).value = 20;
    ptr->show();          // Equivalent to (*ptr).show();

    return 0;
}
\end{lstlisting}

In C++, the \texttt{private} and \texttt{public} keywords define access
control for members within a class. They determine how and where those
members can be accessed outside the class. \texttt{private} members are
accessible only within the class in which they're declared. \texttt{public}
members are accessible from anywhere a member of the class is visible.1

\end{document}

%\begin{center}
%    \begin{forest}
%        [0.INTMAX [1.16 [2.12 [3.10] [3.12]] [3.16 [4.14] [4.16]]][4.INTMAX [5.20 [6.18] [6.20]] [6.INTMAX]]]
%    \end{forest}    
%\end{center}