\section{Access Levels}

\subsection{Public}
Public members are accessible from any part of the program
where the object is visible. They form the interface of the
class. In inheritance, public members remain public when using
public inheritance, though they can become private under private
inheritance.

\begin{lstlisting}[language=C++]
class MyClass {
public:
    int publicVar;
    void publicMethod();
};

int main() {
    MyClass obj;
    obj.publicVar = 42;   // Accessible from anywhere
    obj.publicMethod();   // Accessible from anywhere
    return 0;
}
\end{lstlisting}

\subsection{Protected}
Protected members are accessible within the class that declares
them, as well as in classes derived from it and friend classes.
In public inheritance, protected members remain protected in the
derived class. However, they cannot be accessed from outside the
class hierarchy.

\begin{lstlisting}[language=C++]
class Base {
protected:
    int protectedVar;
public:
    Base() : protectedVar(0) {}
};

class Derived : public Base {
public:
    void modify() {
        protectedVar = 10;  // Accessible here in the derived class
    }
};

int main() {
    Derived d;
    // d.protectedVar = 5; // Error: protectedVar is not accessible outside Base or Derived
    return 0;
}
\end{lstlisting}

\subsection{Private}
Private members are only accessible within the class that declares
them and by its friend functions or classes. They are not directly
accessible in derived classes, although they do exist as part of the
derived object. If a derived class needs to interact with these members,
it must do so through public or protected member functions of the base class.

\begin{lstlisting}[language=C++]
class Base {
private:
    int privateVar;
protected:
    int protectedVar;
public:
    Base() : privateVar(0), protectedVar(0) {}
};

class Derived : public Base {
public:
    void tryAccess() {
        // privateVar = 5;  // Error: privateVar is not accessible here
        protectedVar = 10;  // OK: protectedVar is accessible in the derived class
    }
};

int main() {
    Base b;
    // b.privateVar = 5;  // Error: cannot access private member
    return 0;
}
\end{lstlisting}

% Note on Inheritance Nuances:
% In public inheritance, the access levels of the base class members are preserved:
%
%   - Public members remain public.
%   - Protected members remain protected.
%
% In private inheritance, however, both public and protected members of the base class become private in the derived class:
%
% \begin{lstlisting}[language=C++]
% class Base {
% public:
%     int pub;
% protected:
%     int prot;
% };
%
% class DerivedPublic : public Base {
%     void access() {
%         pub = 1;   // OK: remains public
%         prot = 2;  // OK: remains protected
%     }
% };
%
% class DerivedPrivate : private Base {
%     void access() {
%         pub = 1;   // OK: but pub is now private in DerivedPrivate
%         prot = 2;  // OK: prot is now private in DerivedPrivate
%     }
% };
%
% int main() {
%     DerivedPublic d1;
%     d1.pub = 1;       // OK
%     // d1.prot = 2;   // Error: protected in DerivedPublic
%
%     DerivedPrivate d2;
%     // d2.pub = 1;    // Error: pub is private in DerivedPrivate
%     return 0;
% }
% \end{lstlisting}
%
% This illustrates that while the access specifiers in 
% the base class define member accessibility within the 
% base and derived classes, the type of inheritance can 
% further restrict external access.
