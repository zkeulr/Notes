\section{Inheritance}

The \emph{has-a} relationship describes a class C
using an object of class H to hold information
about class C. For instance, consider a \texttt{Car}
class that has a string \texttt{model} within it.
\texttt{Car} has \texttt{String}.

The \emph{is-a} relationship is when an object
of one class is also an object of another class.
For instance, a \texttt{Student} instance may also be a
\texttt{Person} class. We say that \texttt{Student}
\emph{inherits} or \emph{extends} \texttt{Person}.

\begin{lstlisting}[language=C++, caption={Inheritance}]
    #include <iostream>
    
    // Base class
    class Animal {
    public:
        void eat() {
            std::cout << "This animal eats." << std::endl;
        }
    };
    
    // Derived class inheriting from Animal
    class Dog : public Animal {
    public:
        void bark() {
            std::cout << "The dog barks." << std::endl;
        }
    };
    
    int main() {
        Dog myDog;
        myDog.eat();  // Inherited from Animal
        myDog.bark(); // Defined in Dog
        return 0;
    }
    \end{lstlisting}