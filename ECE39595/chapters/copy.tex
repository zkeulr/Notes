\section{Copy Constructor}

The copy constructor allows you to copy the values of one object
to another. If you have any pass-by-value functions in your class, 
then you'll need to create a copy constructor since pass-by-value 
creates a copy of the object to manipulate within the method. 

\begin{lstlisting}[language=C++,caption=Copy Constructor]
#include <iostream>
#include <cstring>

class Person {
private:
    char* name;
    int age;

public:
    Person(const char* personName, int personAge) {
        name = new char[strlen(personName) + 1]; // Allocate memory
        strcpy(name, personName); // Copy the string
        age = personAge;
    }

    // Copy constructor
    Person(const Person& other) {
        name = new char[strlen(other.name) + 1]; // Allocate new memory
        strcpy(name, other.name); // Copy the name
        age = other.age; // Copy the age
    }
};

int main() {
    Person person1("Alice", 25); // Create the first object
    Person person2 = person1; // Use the copy constructor

    return 0;
}
\end{lstlisting}

We must be careful in the case of self-assignment (\texttt{x = x}), 
especially if anything is being deleted in the copy constructor. 
It's wise to include a check for self assignment and to simply 
return \texttt{*this} if true. 

\begin{lstlisting}[language=C++,caption=Self Assignment Check]
    const MyClass& MyClass::operator=(const MyClass& copyFrom) {
        if(this == &copyFrom) { return *this; }

        // other copying logic
    }
    \end{lstlisting}