\section{Objects and Classes}
A class stores functions (methods) and values for a bunch
of objects you may want to create that follow its blueprint,
like characters in a video game or car models.

\begin{lstlisting}[language=C++,caption=Objects and Classes]
    #include <iostream>

    class Dog {
            public:
            std::string breed;
            std::string name;

            void bark() {
                    std::cout << "Woof!" << std::endl;
                }
        };

    int main() {
            Dog my_dog;
            my_dog.breed = "Labrador";
            my_dog.name = "Buddy";

            Dog your_dog;
            your_dog.breed = "Poodle";
            your_dog.name = "Coco";

            std::cout << my_dog.breed << std::endl;
            my_dog.bark();

            return 0;
        }
\end{lstlisting}

There are many ways to create a new object in C++.

\begin{lstlisting}[language=C++,caption=Alternative Instantiations]
#include "Dog.h"
int main(int argc, char* argv[ ]) {
    Dog* myDog = new Dog("Sam");
    Dog myDog("Sanjay");
    Dog myDog = Dog("Jenna");
    ...
}
\end{lstlisting}

There are also many ways to access values in an object.
One potentially novel way is through the \texttt{->} operator.
It's shorthand for dereferencing a pointer and then accessing
a member of the object it points to.
\begin{lstlisting}[language=C++,caption=-> Operator]
class Example {
public:
    int value;
    void show() {
        std::cout << "Value: " << value << std::endl;
    }
};

int main() {
    Example obj;
    obj.value = 10;

    Example* ptr = &obj;  // Pointer to the object

    // Accessing members using the pointer
    ptr->value = 20;      // Equivalent to (*ptr).value = 20;
    ptr->show();          // Equivalent to (*ptr).show();

    return 0;
}
\end{lstlisting}

In C++, the \texttt{private} and \texttt{public} keywords define access
control for members within a class. They determine how and where those
members can be accessed outside the class. \texttt{private} members are
accessible only within the class in which they're declared. \texttt{public}
members are accessible from anywhere a member of the class is visible.1