\section{Generic Programming}

Generic programming is a style of computer programming in which
algorithms are written in terms of types to-be-specified-later
that are then instantiated when needed for specific types provided
as parameters. This approach allows for code reusability and type safety.
Generic programming is primarily achieved through the use of templates.
Templates allow functions and classes to operate with generic types,
which makes it possible to create a function or class to work with
any data type.

A function template works by defining a pattern for a function that can
operate on any data type. Here is an example of a simple function template
that returns the maximum of two values:

\begin{lstlisting}[language=C++]
template <typename T>
T max(T a, T b) {
    return (a > b) ? a : b;
}
\end{lstlisting}
Class templates allow you to define a class that can operate with
any data type. Here is an example of a simple class template for a
pair of values:

\begin{lstlisting}[language=C++]
template <typename T1, typename T2>
class Pair {
public:
    Pair(T1 first, T2 second) : first_(first), second_(second) {}
    T1 first() const { return first_; }
    T2 second() const { return second_; }
private:
    T1 first_;
    T2 second_;
};
\end{lstlisting}