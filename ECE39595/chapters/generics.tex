\section{Generic Programming}

Generic programming is a style of computer programming in which
algorithms are written in terms of types to-be-specified-later
that are then instantiated when needed for specific types provided
as parameters. This approach allows for code reusability and type safety.
Generic programming is primarily achieved through the use of templates.
Templates allow functions and classes to operate with generic types,
which makes it possible to create a function or class to work with
any data type.

A function template works by defining a pattern for a function that can
operate on any data type. Here is an example of a simple function template
that returns the maximum of two values:

\begin{lstlisting}[language=C++]
template <typename T>
T max(T a, T b) {
    return (a > b) ? a : b;
}
\end{lstlisting}
Class templates allow you to define a class that can operate with
any data type. Here is an example of a simple class template for a
pair of values:

\begin{lstlisting}[language=C++]
template <typename T1, typename T2>
class Pair {
public:
    Pair(T1 first, T2 second) : first_(first), second_(second) {}
    T1 first() const { return first_; }
    T2 second() const { return second_; }
private:
    T1 first_;
    T2 second_;
};
\end{lstlisting}

\subsection{Function Pointers}

Function pointers are variables that store memory addresses of functions. They allow functions to be passed as arguments to other functions, returned from functions, and stored in data structures. In the context of generic programming, function pointers enable writing algorithms that can work with different behaviors specified at runtime.

The syntax for declaring a function pointer involves specifying the return type and parameter types:

\begin{lstlisting}[language=C++]
// Declaring a function pointer
return_type (*pointer_name)(parameter_types);
\end{lstlisting}

Here's a basic example of using function pointers:

\begin{lstlisting}[language=C++]
#include <iostream>

int add(int a, int b) {
    return a + b;
}

int subtract(int a, int b) {
    return a - b;
}
int main() {
    // Declare function pointer
    int (*operation)(int, int);
    
    // Assign function pointer to add
    operation = add;
    std::cout << "Result of add: " << operation(5, 3) << std::endl;
    
    // Reassign function pointer to subtract
    operation = subtract;
    std::cout << "Result of subtract: " << operation(5, 3) << std::endl;
    
    return 0;
}
\end{lstlisting}

Function pointers can be combined with templates for powerful generic programming:

\begin{lstlisting}[language=C++]
template <typename T>
T processValues(T a, T b, T (*operation)(T, T)) {
    return operation(a, b);
}

template <typename T>
T multiply(T a, T b) {
    return a * b;
}
int main() {
    // Use the generic function with different types
    int intResult = processValues<int>(5, 3, multiply);
    double doubleResult = processValues<double>(2.5, 1.5, multiply);
    
    std::cout << "Int result: " << intResult << std::endl;
    std::cout << "Double result: " << doubleResult << std::endl;
    
    return 0;
}
\end{lstlisting}

Modern C++ provides std::function from the <functional> header, which offers a more flexible alternative to traditional function pointers:

\begin{lstlisting}[language=C++]
#include <iostream>
#include <functional>

template <typename T>
T processValues(T a, T b, std::function<T(T, T)> operation) {
    return operation(a, b);
}

int main() {
    // Using lambda expressions
    auto divide = [](double a, double b) { return a / b; };
    
    double result = processValues<double>(10.0, 2.0, divide);
    std::cout << "Result: " << result << std::endl;
    
    return 0;
}
\end{lstlisting}