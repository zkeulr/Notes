\section{Introduction}
Let's examine some basic C++ programs.
\begin{lstlisting}[language=C++,caption=Hello World]
    #include <iostream>

    int main() {
        std::cout << "Hello World!";
        return 0;
    }
\end{lstlisting}

\begin{lstlisting}[language=C++,caption=User Input]
    #include <iostream>

    int main() {
        double n;
        int i;

        std::cout << "Enter float: ";
        std::cin >> n;

        std::cout << "Enter integer: ";
        std::cin >> i;

        return 0;
    }
\end{lstlisting}

The \texttt{stds} you're seeing all over the place refer not
to a frat party but the standard namespace. It holds
useful objects like standard in ("\texttt{cin}") and standard
out ("\texttt{cout}").

In C, we use \texttt{malloc} and \texttt{free} to allocate memory. In C++,
the equivalent operations are \texttt{new} and \texttt{delete}.

\begin{lstlisting}[language=C++,caption=Free and Delete]
    void CorrectUsage(){
        int *ptr = new int[3];
        int *ptr1 = new int;
        ptr[0] = 1;
        ptr[1] = 2;
        ptr[2] = 3;
        *ptr1 = 5;
        delete ptr1;
        delete [] ptr;        
\end{lstlisting}
The reason for this is using \texttt{new} and \texttt{free} calls
an object's constructor and destructors, which are defined to properly
delete the object. Technically, \texttt{malloc} and \texttt{free} are
both present in C++, but they won't trigger the constructors and
destructors and should be avoided. The choice to leave these functions in
was made to improve compatibility with C, which is a theme the reader
may notice in C++'s many possible ways to do the same thing.

C++ filenames are terminated with a \texttt{.cpp} or \texttt{.cc}
extension, like \texttt{example.cpp} or \texttt{example.cc}