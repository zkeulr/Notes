\section{Threading}

A \emph{process} is an independent program in execution, with its own 
memory space, while a \emph{thread} is a smaller unit of execution 
within a process that shares the same memory space as other threads 
in the same process. Threads are lightweight and allow multiple tasks 
to run concurrently within a single process.

Threads are commonly used to perform tasks in parallel, such as handling 
multiple client requests in a server or performing background computations 
while keeping the main application responsive. However, threads don't 
return anything, so variables that need modified should be passed by reference. 

\subsection{Creating Threads in C++}
C++ provides support for multithreading through the \texttt{<thread>} header. 
A thread can be created by instantiating a \texttt{std::thread} object 
and passing a callable (such as a function, lambda, or functor) to its 
constructor.

Here is an example of creating and joining threads:

\begin{lstlisting}[language=C++]
#include <iostream>
#include <thread>

// Function to be executed by a thread
void printMessage(const std::string& message) {
    std::cout << "Thread says: " << message << std::endl;
}

int main() {
    // Create a thread and pass a function to it
    std::thread t1(printMessage, "Hello from thread!");

    // Wait for the thread to finish
    t1.join();

    std::cout << "Main thread finished." << std::endl;
    return 0;
}
\end{lstlisting}

In this example, a thread \texttt{t1} is created to execute the \texttt{printMessage} function.
The \texttt{join()} method ensures that the main thread waits for \texttt{t1} to complete before proceeding.

\subsection{Using Lambda Functions with Threads}
Lambda functions are often used with threads for simplicity and 
flexibility. Here's an example:

\begin{lstlisting}[language=C++]
#include <iostream>
#include <thread>

int main() {
    // Create a thread using a lambda function
    std::thread t([]() {
        std::cout << "Hello from a lambda thread!" << std::endl;
    });

    // Wait for the thread to finish
    t.join();

    std::cout << "Main thread finished." << std::endl;
    return 0;
}
\end{lstlisting}

\subsection{Thread Synchronization}
When multiple threads access shared resources, synchronization is required to avoid race conditions. C++ provides synchronization primitives like \texttt{std::mutex} to protect shared data.

Here is an example of using a mutex to synchronize threads:

\begin{lstlisting}[language=C++]
#include <iostream>
#include <thread>
#include <mutex>

std::mutex mtx; // Mutex for synchronization
int sharedCounter = 0;

void incrementCounter() {
    for (int i = 0; i < 1000; ++i) {
        std::lock_guard<std::mutex> lock(mtx); // Lock the mutex
        ++sharedCounter;
    }
}

int main() {
    std::thread t1(incrementCounter);
    std::thread t2(incrementCounter);

    t1.join();
    t2.join();

    std::cout << "Final counter value: " << sharedCounter << std::endl;
    return 0;
}
\end{lstlisting}

In this example, \texttt{std::mutex} ensures that only one thread modifies \texttt{sharedCounter} at a time.
\texttt{std::lock\_guard} automatically locks and unlocks the mutex within its scope.

\subsection{Detaching Threads}
A thread can be detached to run independently of the main thread. 
However, care must be taken to ensure that the thread does not access 
resources that may go out of scope.

\begin{lstlisting}[language=C++]
#include <iostream>
#include <thread>
#include <chrono>

void backgroundTask() {
    std::this_thread::sleep_for(std::chrono::seconds(2));
    std::cout << "Background task completed." << std::endl;
}

int main() {
    std::thread t(backgroundTask);
    t.detach(); // Detach the thread

    std::cout << "Main thread continues execution." << std::endl;
    std::this_thread::sleep_for(std::chrono::seconds(3)); // Ensure main thread lives long enough
    return 0;
}
\end{lstlisting}

In this example, the \texttt{detach()} method allows the thread to run independently.
The main thread ensures it lives long enough for the detached thread to complete.