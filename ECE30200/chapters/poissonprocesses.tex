\section{Poisson Processes}
If
\begin{equation}
    S_n = \sum_{i=1}^{n}X_i
\end{equation}
where $X_i \sim \text{Exponential}(\lambda)$
then
\begin{equation}
    S_n \sim \text{Gamma}(n, \lambda)
\end{equation}
and
\begin{equation}
    f_{S_n}(s) = \frac{\lambda^n s^{n-1} e^{-\lambda s}}{(n-1)!}
\end{equation}

If
\begin{equation}
    S_n = \sum_{i=1}^{n}X_i
\end{equation}
where $X_i \sim \text{Poisson}(\lambda_i)$
then
\begin{equation}
    S_n \sim \text{Poisson}(\sum_{i=1}^{n}, \lambda_i)
\end{equation}

When we say that exponential random variables
are memoryless, we mean that if
\begin{equation}
    X \sim \text{Exponential}(\lambda)
\end{equation}
then
\begin{equation}
    P(X > t + s | X > s) = P(X > t)
\end{equation}

A \emph{Poisson process} is a mathematical model for a
sequence of events that occur randomly in time (or space)
but with a constant average rate.
It can be defined thus:
If $X_n \sim \text{Exponential}(\lambda)$ for
$n = 1, 2, 3, \dots$ then
\begin{equation}
    N(t) = \max \left\{n: \sum_{i=1}^{n} X_i \leq t\right\}
\end{equation}
is called a Poisson process with rate $\lambda$.
In more words, if the interarrival times follow
an exponential distribution with rate $\lambda$,
then the number of arrivals by time $t$ is called
a Poisson process with rate $\lambda$.

The number of arrivals by time $s$ follows
a Poisson distribution
\begin{equation}
    N(s) \sim \text{Poisson}(\lambda s)
\end{equation}

A Poisson process has two important
properties:
\begin{equation}
    N(t+s) - N(s) \sim \text{Poisson}(\lambda t)
\end{equation}
and
$N(t_1) - N(t_0), \dots, N(t_n) - N(t_{n-1})$ are
independent.

This is rather surprising. It is surprising that
if $N(t)$ is a Poisson process with rate $\lambda$
then $\tilde{N}(t) = N(t + r) - N(r)$ is a Poisson
process with rate $\lambda$ and is independent of $N(r)$.