\subsection{Probability Properties}
\begin{equation}
    P\left[A \cup B\right] = P\left[A\right] + P\left[B\right] - P\left[A \cap B\right]
\end{equation}
\begin{equation}
    P\left[A \cup B\right] \leq P\left[A\right] + P\left[B\right]
\end{equation}
\begin{equation}
    A \subseteq B \implies P\left[A\right] \leq P\left[B\right]
\end{equation}
\begin{equation}
    P\left[A | B\right] = \frac{P[A \cap B]}{P[B]}
\end{equation}

Outcomes are statistically \emph{independent} if
$P(A|B) = P(A)$ (assuming P(B) > 0), or
equivalently $P(A \cap B) = P(A)P(B)$.

A \emph{random variable} $X$ is a function $X : \Omega \implies \Re$
that maps an outcome $\epsilon \in \Omega$ to a number $X(\epsilon)$ on the real line.

\emph{Bayes Theorem} states that for any two events $A$ and $B$ such that $P[A] > 0$ and
$P[B] > 0$,
\begin{equation}
    P[A|B] = \frac{P[B|A]P[A]}{P[B]}
\end{equation}

The \emph{Law of Total Probability} states that if
$\{A_1, A_2, \dots, A_n\}$ is a partition of $\Omega$,
then for any $B \subseteq \Omega$,
\begin{equation}
    P[B] = \sum_{i = 1}^{n} P[B|A_i]P[A_i]
\end{equation}