\section{Background}

The following formulas will be instrumental and may be familar.

\subsection{Series}
\begin{equation}
    \sum_{k=0}^{n} r^k = \frac{1 - r^{n + 1}}{1 - r}
\end{equation}
\begin{equation}
    \sum_{n=1}^{\infty} \frac{1}{n^2} = \frac{\pi^2}{6}
\end{equation}
\begin{equation}
    \sum_{k=1}^{\infty}kr^{k - 1} = \frac{1}{(1 - r)^2}
\end{equation}

\subsection{Combinatorics}
\begin{equation}
    {n\choose k} = \frac{n!}{k!(n - k)!}
\end{equation}
\begin{equation}
    (a + b)^n = \sum_{k=0}^{n} {n\choose k} a^{n - k}b^k
\end{equation}
\begin{equation}
    {n\choose k} + {n\choose k-1} = {n + 1\choose k}
\end{equation}
\begin{equation}
    P(n, k) = \frac{n!}{(n-k)!}
\end{equation}
where $P(n, k)$ is the number of ways to arrange $k$ objects out of $n$ (permutations).
\begin{equation}
    C(n, k) = {n\choose k} = \frac{n!}{k!(n-k)!}
\end{equation}
where $C(n, k)$ is the number of ways to choose $k$ objects out of $n$ (combinations).

\subsection{Approximations}
\begin{align}
    f(x) & = f(a) + f'(a)(x - a) + \frac{f''(a)}{2!}(x - a)^2 + \dots \\
         & = \sum_{n=0}^{\infty} \frac{f^{(n)}(a)}{n!}(x - a)^n
\end{align}
\begin{align}
    1 + x + \frac{x^2}{2!} + \frac{x^3}{3!} + \dots & = \sum_{k=0}^{\infty} \frac{x^k}{k!} \\
                                                    & = e^x
\end{align}
\begin{align}
    \sin(x) & = x - \frac{x^3}{3!} + \frac{x^5}{5!} - \frac{x^7}{7!} + \dots \\
            & = \sum_{n=0}^{\infty} (-1)^n \frac{x^{2n+1}}{(2n+1)!}
\end{align}
\begin{align}
    \cos(x) & = 1 - \frac{x^2}{2!} + \frac{x^4}{4!} - \frac{x^6}{6!} + \dots \\
            & = \sum_{n=0}^{\infty} (-1)^n \frac{x^{2n}}{(2n)!}
\end{align}
\begin{align}
    \ln(1 + x) & = x - \frac{x^2}{2} + \frac{x^3}{3} - \frac{x^4}{4} + \dots \\
               & = \sum_{n=1}^{\infty} (-1)^{n+1} \frac{x^n}{n}
\end{align}

\subsection{Calculus}
\begin{equation}
    \frac{d}{dx} \int_{a}^{x} f(t)\,dt = f(x)
\end{equation}
\begin{equation}
    \int_{a}^{b} f'(x)\,dx = f(b) - f(a)
\end{equation}
\begin{equation}
    \int f(g(x))g'(x)\,dx = \int f(u)\,du
\end{equation}
\begin{equation}
    \int u\,dv = uv - \int v\,du
\end{equation}
\begin{equation}
    \int \frac{1}{(x-a)(x-b)}\,dx = \frac{1}{b-a} \ln\left|\frac{x-a}{x-b}\right| + C
\end{equation}

\subsection{Linear Algebra}
\begin{equation}
    \vec{y} = \beta_1 \vec{x_1} + \beta_2 \vec{x_2} + \cdots + \beta_N \vec{x_N}
\end{equation}
\begin{align}
    \langle \vec{a}, \vec{b} \rangle & = \vec{a}\vec{b}^{T}     \\
                                     & = \sum_{i=1}^{n} a_i b_i
\end{align}
where $\langle \vec{a}, \vec{b} \rangle$ denotes the inner product of vectors $\vec{a}$ and $\vec{b}$.
\begin{equation}
    \|\vec{x}\|_p = \left( \sum_{i=1}^{n} |x_i|^p \right)^{1/p}
\end{equation}
where $\|\vec{x}\|_p$ is the $p$-norm (or $\ell_p$-norm) of vector $\vec{x}$.
\begin{equation}
    \cos(\theta) = \frac{\langle \vec{a}, \vec{b} \rangle}{\|\vec{a}\|_2 \|\vec{b}\|_2}
\end{equation}
where $\theta$ is the angle between vectors $\vec{a}$ and $\vec{b}$.
\begin{equation}
    \hat{\beta} = (\mathbf{X}^T \mathbf{X})^{-1} \mathbf{X}^T \vec{y}
\end{equation}
where $\hat{\beta}$ is the vector of least squares coefficients, $\mathbf{X}$ is the data matrix, and $\vec{y}$ is the target vector

\subsection{Set Theory}
Some important properties of set operations are:
\begin{itemize}
    \item \textbf{Commutativity:}
          \begin{align}
              A \cup B & = B \cup A \\
              A \cap B & = B \cap A
          \end{align}
    \item \textbf{Associativity:}
          \begin{align}
              (A \cup B) \cup C & = A \cup (B \cup C) \\
              (A \cap B) \cap C & = A \cap (B \cap C)
          \end{align}
    \item \textbf{Distributivity:}
          \begin{align}
              A \cup (B \cap C) & = (A \cup B) \cap (A \cup C) \\
              A \cap (B \cup C) & = (A \cap B) \cup (A \cap C)
          \end{align}
    \item \textbf{Identity:}
          \begin{align}
              A \cup \emptyset & = A \\
              A \cap \Omega    & = A
          \end{align}
    \item \textbf{Complement:}
          \begin{align}
              A \cup A^c & = \Omega    \\
              A \cap A^c & = \emptyset
          \end{align}
\end{itemize}

\subsection{Probability Laws}
A probability law must satisfy three axioms:
\begin{enumerate}
    \item Non-negativity: $P(A) \geq 0 \forall A \in F$
    \item Normalization: $P(\Omega) = 1$
    \item Additivity: For any disjoint subsets $\{A_1, A_2, \dots\}$,
          it holds that
          \[P\left[\bigcup_{n=1}^{\infty}A_n\right] = \sum_{n=1}^{\infty}P\left[A_n\right]\]
\end{enumerate}

\subsection{Probability Properties}
\begin{equation}
    P\left[A \cup B\right] = P\left[A\right] + P\left[B\right] - P\left[A \cap B\right]
\end{equation}
\begin{equation}
    P\left[A \cup B\right] \leq P\left[A\right] + P\left[B\right]
\end{equation}
\begin{equation}
    A \subseteq B \implies P\left[A\right] \leq P\left[B\right]
\end{equation}