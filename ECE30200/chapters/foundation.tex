\section{Formal Definitions}

\subsection{Outcomes}
An \emph{outcome} is the result of some \emph{experiment}.
If that experiment is flipping a coin, the outcome is either
heads or tails. We could express the outcome of heads as $H$,
and the outcome of tails as $T$. The set of all possible
outcomes for an experiment is known as a sample space and is
denoted by $\Omega$. In this case $\Omega = \{H, T\}$.

\subsection{Events}
An \emph{event} $F$ is a subset of the probability space $\Omega$.
The formal definitions of probability are expressed with set
notation. So the event where we have neither heads nor tails is
written as $\{\}$. The event of heads could be expressed as
$\{H\}$, and the event of tails could be expressed as $\{T\}$.
The event of either heads or tails is $\{\Omega\}$.

\subsection{Probability Laws}
A \emph{probability law} is a function $P$ that maps an event $A$
to a real number in $[0, 1]$. For the coin example, the probability
law might be $P(\{\}) = 0$, $P(\{H\}) = 0.5$, $P(\{T\}) = 0.5$, and
$P(\{\Omega\}) = 1$. A probability law must satisfy three axioms:
\begin{enumerate}
    \item Non-negativity: $P(A) \geq 0 \forall A \in F$
    \item Normalization: $P(\Omega) = 1$
    \item Additivity: For any disjoint subsets $\{A_1, A_2, \dots\}$,
          it holds that
          \[P\left[\bigcup_{n=1}^{\infty}A_n\right] = \sum_{n=1}^{\infty}P\left[A_n\right]\]
\end{enumerate}

\subsection{Probability Space}
A probability space is a triplet $\Omega, F, P$.

\subsection{Probability Properties}
\begin{equation}
    P\left[A \cup B\right] = P\left[A\right] + P\left[B\right] - P\left[A \cap B\right]
\end{equation}
\begin{equation}
    P\left[A \cup B\right] \leq P\left[A\right] + P\left[B\right]
\end{equation}
\begin{equation}
    A \subseteq B \implies P\left[A\right] \leq P\left[B\right]
\end{equation}
\begin{equation}
    P\left[A | B\right] = \frac{P[A \cap B]}{P[B]}
\end{equation}
