\section{Central Limit Theorem}
Let $\bar{X}_N$ be the sample average,
and let
\begin{equation}
    Z_N = \sqrt{N} \left(\frac{\bar{X}_N-\mu}{\sigma}\right)
\end{equation}
be the normalized variable. The \emph{central limit theorem}
is: the CDF of $Z_N$ converges pointwise to the
CDF of Gaussian(0,1). The choice of language
is extremely careful here. We are not saying that the
PDF of $Z_N$ converges to the PDF of a Gaussian, nor that
the random variable $Z_N$ converges to a Gaussian random variable.
Formally, we write
\begin{equation}
    \lim_{N\rightarrow \infty} F_{\bar{Z_N}(z)} = F_Z(z)
\end{equation}

In practice, what this means is that if $X_1, X_2, \dots, X_N$
are random variables with means $\mu_1, \mu_2, \dots, \mu_N$ and
variances $\sigma_1^2, \sigma_2^2, \dots, \sigma_N^2$, then
\begin{equation}
    \frac{1}{N} \sum_{i=1}^{N} X_i \sim \mathcal{N}(\frac{1}{N} \sum_{i=1}^{N} \mu_i, \frac{1}{N^2} \sum_{i=1}^{N} \sigma_i^2).
\end{equation}
If $\{X_i\}$ are i.i.d. with mean $\mu$ and variance $\sigma^2$, then
\begin{equation}
    \frac{1}{N} \sum_{i=1}^{N} X_i \sim \mathcal{N}\left(\mu, \frac{\sigma^2}{N}\right)
\end{equation}