\documentclass[nobib]{tufte-handout}

%\\geometry{showframe}% for debugging purposes -- displays the margins

\newcommand{\bra}[1]{\left(#1\right)}
\usepackage{clrscode3e}
\usepackage{amssymb}
\usepackage{hyperref}
\usepackage[activate={true,nocompatibility},final,tracking=true,kerning=true,spacing=true,factor=1100,stretch=10,shrink=10]{microtype}
\usepackage{color}
\usepackage{steinmetz}
% Fixes captions and images being cut off
\usepackage{marginfix}
\usepackage{array}
\usepackage{tikz}
\usepackage{amsmath,amsthm}
\usetikzlibrary{shapes}
\usetikzlibrary{positioning}
\usepackage{listings}
\usepackage{caption}
\DeclareCaptionFont{white}{\color{white}}
\DeclareCaptionFormat{listing}{\colorbox{gray}{\parbox{\textwidth}{#1#2#3}}}
\captionsetup[lstlisting]{format=listing,labelfont=white,textfont=white}

% Set up the images/graphics package
\usepackage{graphicx}
\setkeys{Gin}{width=\linewidth,totalheight=\textheight,keepaspectratio}
\graphicspath{{.}}

\title{ECE 20875: Python for Data Science}
\author[Zeke Ulrich]{Zeke Ulrich}
\date{\today}  % if the \date{} command is left out, the current date will be used

% The following package makes prettier tables.  We're all about the bling!
\usepackage{booktabs}

% The units package provides nice, non-stacked fractions and better spacing
% for units.
\usepackage{units}

% The fancyvrb package lets us customize the formatting of verbatim
% environments.  We use a slightly smaller font.
\usepackage{fancyvrb}
\fvset{fontsize=\normalsize}

% Small sections of multiple columns
\usepackage{multicol}

% For finite state machines 
\usetikzlibrary{automata} % Import library for drawing automata
\usetikzlibrary{positioning} % ...positioning nodes
\usetikzlibrary{arrows} % ...customizing arrows
\tikzset{node distance=2.5cm, % Minimum distance between two nodes. Change if necessary.
    every state/.style={ % Sets the properties for each state
    semithick,
    fill=gray!10},
    initial text={}, % No label on start arrow
    double distance=2pt, % Adjust appearance of accept states
    every edge/.style={ % Sets the properties for each transition
    draw,
    ->,>=stealth', % Makes edges directed with bold arrowheads
    auto,
    semithick}}
\let\epsilon\varepsilon

% These commands are used to pretty-print LaTeX commands
\newcommand{\doccmd}[1]{\texttt{\textbackslash#1}}% command name -- adds backslash automatically
\newcommand{\docopt}[1]{\ensuremath{\langle}\textrm{\textit{#1}}\ensuremath{\rangle}}% optional command argument
\newcommand{\docarg}[1]{\textrm{\textit{#1}}}% (required) command argument
\newenvironment{docspec}{\begin{quote}\noindent}{\end{quote}}% command specification environment
\newcommand{\docenv}[1]{\textsf{#1}}% environment name
\newcommand{\docpkg}[1]{\texttt{#1}}% package name
\newcommand{\doccls}[1]{\texttt{#1}}% document class name
\newcommand{\docclsopt}[1]{\texttt{#1}}% document class option name

% Define a custom command for definitions and biconditional
\newcommand{\defn}[2]{\noindent\textbf{#1}:\ #2}
\let\biconditional\leftrightarrow

% Define graphics path
\graphicspath{ {./images/} }

\begin{document}

\maketitle

\begin{abstract}
    Lecture notes for Purdue's ECE 20875.
\end{abstract}

\tableofcontents

\section{Course Introduction}

This course will introduce Python programming 
to students through data science problems. 
Students will learn Python concepts as well 
as introductory data science topics, 
and will use their knowledge of Python 
(and prior programming experience) to implement 
data analyses. \\~\\ Learning Objectives:
\begin{enumerate}
    \item Understanding of regular expressions.
    \item Use Python to write data analyses.
    \item Explain when data analyses are appropriate. 
    \item Explain the results of data analyses. 
    \item Incorporate classes in Python. 
    \item Incorporate associative arrays in programs. 
\end{enumerate}

\pagebreak

\section{Introduction to data science}
Data is defined as quantified information. 
By analyzing data we can make informed decisions 
and take actions likely to achieve our goals. Recently, humans have been 
able to record a lot more data and create machines to use it. 
Data science collects data from a variety of sources, 
puts them in a consistent format, makes observations, 
visualizes trends, interprets, and most importantly predicts. 
Data science is pervasive and examples 
can be found in fields spanning healthcare, 
education, finance, and shopping logistics. 

Python is the industry standard for data science 
and has many useful, expansive 
open source libraries for data processing 
like pandas, numpy, and keras. 
\end{document}
