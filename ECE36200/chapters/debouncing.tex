\section{Debouncing}

When a mechanical button is pressed, there is a
period where the signal is neither high nor low,
because of vibrations in the conducting plate it
connects or other non-ideal physical effects. This
is known as \emph{bouncing}, and it's rectified by
adding a Schmitt trigger and RC circuit to smooth
out the signal.

However, in real world systems we rarely have just one
button. We often have a matrix input, like a keypad.
Our Schmitt trigger setup doesn't work here, but
luckily a software solution exists. Set up the CPU
connected to the keypad to scan each key for being pressed.
It doesn't need to check constantly, just often enough that
it will catch a button press.

The work of scanning can be done incrementally using a timer interrupt.
On each interrupt, the ISR will:
\begin{itemize}
    \item read all the columns
    \item put the value read for each key of the current row into its own history byte
    \item turn off the voltage for the row
    \item turn on the voltage for the next row (for the next ISR invocation)
    \item return
\end{itemize}

The keys on a keypad can still bounce. Pressing and letting go of a
button may look something like \texttt{00000001001011111...1111101000000}.
In this example, the button bounces on press and release. However, the
history byte eventually stabilizes and is full of entirely \texttt{1}s
or \texttt{0}s. To detect a press or release, we search all the history
bytes that represent the keys. The first time we detect a change, like in
\texttt{00000001}, we say that is the start of a press or release. We say
the press or release is done when the history byte queue is back to only
one number.

We want to scan the keys faster than they can be pressed or released, but
slower than the total bounce time for any key. Say a button can bounce for
10ms, and we scan one row of the 4-row keypad every 1ms. Then when you
press the button, the instance you read a one, you do not know if the
input is stabilized. It could still bounce, since it has been 0ms.
The second time you read a bit, it could still be bouncing, since 4ms is
within the 10ms bounce time. The third time you read a bit, it could still
be bouncing, since it has only been 8ms. By the fourth bit you read, you are
certain it is a one.

