\section{Single-cycle Processor}

We have used the term \emph{processor} without definition thus far,
hoping that the name would convey the idea of a black box performing
operation on some input and return some output. If we were to formulate
it more concretely, we might call a processor a digital component
that performs operations on an external data source, usually memory or
some other data stream. When implementing a processor
there is freedom to define the clocks per instruction (CPI) and
the length of the clock cycle (cycle time), while the ISA and compiler
determine the number of instructions in a program (instruction count)
and performance of a program.

We assume that one entire instruction is performed per clock cycle.
The general instruction pipeline follows and should be internalized,
as it comes up repeatedly:
\begin{enumerate}
    \item Fetch
    \item Decode operands
    \item Execute
    \item Memory
    \item Writeback
\end{enumerate}

Fetch instructions fetch the instruction, then updates the PC. The PC is updated
at the end of every cycle.

ALU instructions make use of the main ALU to perform arithmetic and logic
operations on data stored in registers. These are R-type instructions.

Load instructions move data from the memory to the register file.
There are I-type instructions.

Store instructions move data from the register file to the memory. These
are S-type instructions.

Conditional branch instructions like \texttt{beq} calculate which branch,
or memory address, to jump to as a result of an operation. They compute
the branch condition and branch target. These are SB-type instructions.

A useful way to represent steps and identify required datapath elements
is \emph{register transfer language} (RTL), which is very close to
assembly and shows the path data takes through the unit
in Figure \ref{fig:datapath}.

\begin{figure}
    \includegraphics{images/datapath.png}
    \caption{Datapath}
    \label{fig:datapath}
\end{figure}

For instance, consider the RTL for a general ALU instruction,
\texttt{R[rd] <- R[rs1] op R[rs2]}. \texttt{op} is an arithmetic
operation like \texttt{+} or \texttt{-}. This describes that
\texttt{Ra}, \texttt{Rb}, \texttt{Rw} come from the instruction's
\texttt{rs1}, \texttt{rs2}, and \texttt{rd} fields. The
ALU Operation and RegWr in Figure \ref{fig:datapath} control
logic after decoding the instruction.