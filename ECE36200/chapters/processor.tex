\section{Single-cycle Processor}

We have used the term \emph{processor} without definition thus far,
hoping that the name would convey the idea of a black box performing
operation on some input and return some output. If we were to formulate
it more concretely, we might call a processor a digital component
that performs operations on an external data source, usually memory or
some other data stream. When implementing a processor
there is freedom to define the clocks per instruction (CPI) and
the length of the clock cycle (cycle time), while the ISA and compiler
determine the number of instructions in a program (instruction count)
and performance of a program.

We assume that one entire instruction is performed per clock cycle.
The general instruction pipeline follows and should be internalized,
as it comes up repeatedly:
\begin{enumerate}
    \item Fetch
    \item Decode operands
    \item Execute
    \item Memory
    \item Writeback
\end{enumerate}

A useful way to represent steps and identify required datapath elements
is \emph{register transfer language} (RTL), which is very close to
assembly.