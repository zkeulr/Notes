\section{Direct Memory Access}

Moving memory from one location to another makes poor use of your CPU.
Moving data is so common that we build a co-processor called a
\emph{Direct Memory Access} (DMA)
controller to do this without our CPU.
\marginnote{DMA is also useful for allowing peripherals to
    use memory independent of the CPU}

The workflow for reading data from a peripheral with a CPU is:
\begin{enumerate}
    \item Copy data from peripheral data register to buffer in memory
    \item Copy data from buffer in memory to CPU
\end{enumerate}

A DMA can just copy data directly from the register to the buffer. There
are two kinds:
\begin{itemize}
    \item Flow-through: DMA controller is used as an
          intermediate buffer (useful if we
          need to change the size of data type)
    \item Fly-by: DMA controller only sets up the bus
          between the source and destination
\end{itemize}

A DMA sits on the bus, stores a list of source/destination
pairs being serviced, and is triggered by software, peripherals,
timers, or other interrupts.