\section{Serial Peripheral Interface}

MCUs often need to communicate with
another MCU or peripheral, on the
same board or across boards. Parallel
interfaces can send and receive multiple
bits at the same time. Serial interfaces,
however, send and receive one bit at
a time.

It may seem like parallel is the obvious
choice, but of course we wouldn't
introduce serial interfaces if they didn't
have some advantages. Parallel interfaces
require more pins, wires and larger
PCB footprint. Wires next to one another
leads to crosstalk, electromagnetic
interference.
\marginnote{Cross-talk can be mitigated
    with twisted wires, but not eliminated}
In practice systems tend to use serial
interfaces. Since such interfaces send
and receive only one bit at a time,
there can either be one wire for both
directions or one wire in each direction.

There are two kinds of peripherals,
\emph{synchronous} and \emph{asynchronous}.
Synchronous protocols have a clock
that dictates the send/receive rate.
Examples include Serial Peripheral Interface
(SPI) and Inter-Integrated Circuit (I2C).
Asynchronous protocols have no clock.
The most famous example of an asynchronous
protocol is Universal Asynchronous
Receiver and Transmitter.

SPI is one of the most common serial
interfaces. It operates in Master/Slave
mode. There must be at least one master,
while there can be multiple slaves. The
master defines the clock and delivers a
clock signal to each data recipient.
It operates by shifting in bits each
clock cycle.

API devices are designated as masters
or slaves. A master device initiates
all data transfer operations. It drives
the clock pin and data pin. It also
drives slave select pins if slaves are
selectively enabled. A slave device
responds to transfer operations. Signals
are named according to the devices' standpoint.
E.g. MOSI: master out, slave in is the
data transmitted by a master device
and received by a slave device.

What does this look like in practice? Say
we wish to send \texttt{0xB1}. Each rising
clock edge latches one bit into the slave.
First, the master selects the slave by asserting
the slave select signal. We send 8 bits,
so there are 8 clock pulses. Finally, one
each clock pulse, one bit is transferred.

The master-slave terminology is being
phased out in industry. Updated datasheets
might refer to SPI pins with terminology such as
\begin{table}[h!]
    \centering
    \begin{tabular}{|c|c|}
        \hline
        \textbf{Legacy Pin Name}    & \textbf{Updated Pin Name}                    \\ \hline
        MOSI (Master Out, Slave In) & SDO (Serial Data Out), TX/TXD (Transmission) \\ \hline
        MISO (Master In, Slave Out) & SDI (Serial Data In), RX/RXD (Reception)     \\ \hline
        SCK (Serial Clock)          & SCK (Serial Clock)                           \\ \hline
        NSS (Negative Slave Select) & CS (Chip Select)                             \\ \hline
    \end{tabular}
    \caption{SPI Pin Names}
    \label{tab:spi_pin_names}
\end{table}

The API clock doesn't need to be perfectly
periodic and continuous. As long as it's not
too fast, it can pause as needed since
the slave takes the data bits only
at clock pulses anyway.

SPI supports multiple slaves, which can share
MOSI, MISO, and clock lines. However,
each slave needs its own select signal line.
The slave selected drives the MISO line. No
more than one slave may be selected at a
time. SPI also supports more than one
master. It is an error for multiple master
devices to assert CS at the same time.
Some coordination is needed to ensure
that one master device reads/writes at any time.

SPI is supported by LCD displays,
OLED displays, SD card interfaces, and
hundreds of other devices.