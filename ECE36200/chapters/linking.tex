\section{Linking}

With a statically linked C program,
the steps to turn source code into
machine code are:
\begin{itemize}
    \item Compiler turns program into assembly
    \item Assembler turns assembly into machine language
    \item Linker includes machine language library routine
    \item Loader loads complete machine language into memory
\end{itemize}

The assembler (or compiler) provides information
for building a complete program from the pieces.
\begin{itemize}
    \item Header: describes contents of object module
    \item Text segment: translated instructions
    \item Static data segment: data allocated for the life of the program (global vars)
    \item Relocation info: for contents that depend on absolute location of loaded program
    \item Symbol table: global definitions and external refs
    \item Debug info: for associating with source code
\end{itemize}

Linking object modules, such as the output of an
assembler, produces an executable image (\texttt{.elf},
\texttt{*.img}). It merges segments, resolves labels, and
patches location-dependent and external references.
For example, \texttt{printf} is a precompiled module and
the linker will link it to any calls in code that a
user compiles.

The loader loads from image file on disk into memory. If a
Linux user types \texttt{ls}, the loader will read \texttt{ls}'s,
header to determine segment sizes, create virtual address space,
copy text and initialized data into memory, set up arguments
on stack, initialize registers, and jump to a startup routine.

\emph{Dynamic linking} only links/loads a library procedure when
called. For instance, if a call to \texttt{printf} is in code
but is never accessed, then the relatively large \texttt{printf}
library is never used and so the executable size is smaller. The
downside is that executables take longer to execute, since
external libraries have to be reached.

