\section{Digital-to-analog Converter}

A digital-to-analog converter (DAC) converts digital data
into a voltage signal.
\begin{equation}
    \text{DAC}_{\text{output}} = V_{\text{ref}} \frac{\text{Digital Value}}{2^N - 1}
\end{equation}
$N$ is the \emph{resolution} of the DAC, and $2^N$ is the number of voltage levels
the DAC can generate.

Applications of DACs include digital audio, video display, and waveform generation.

A 5-bit DAC looks like Figure \ref{fig:dacfivebit}.

\begin{figure}
    \begin{circuitikz}[american]
        \def\N{5} % change this number for N-bit DAC

        \draw (0,0) node[left]{$V_{out}$} coordinate(out);

        \foreach \i [evaluate=\i as \bit using int(\N-1-\i)] in {0,...,4} {
                \draw ( \i*2,0) to[R,l=$R$] ( {(\i+1)*2},0);
                \draw ( {(\i+1)*2},0)
                to[R,l=$2R$] ++(0,2) node[above]{$x_{\bit}$};
                \coordinate (node\i) at ({(\i+1)*2},0);
            }

        \draw (node4) to[R,l=$2R$] ++(0,-2) node[ground]{};
    \end{circuitikz}
    \caption{5-bit DAC}
    \label{fig:dacfivebit}
\end{figure}

The DAC conversion is nearly instantaneous. Some settling time is required
due to capacitance and inductance in the circuit. No iteration or
convergence is needed. However, resistor tolerance is a limitation.
The higher $N$ of the DAC, the more precise the resistors must be.

In place of a resistor ladder, a $2^N$-to-1 mux can be used to select
one voltage level instead.