\section{Signed Arithmetic}

Bits in memory don't inherently represent anything.
What a given value in memory means depends on how
it is interpreted: signed, unsigned, floating point,
little-endian, big-endian.

\subsection{Basic Arithmetic}

The most common implementation of signed integers
is 2's complement, because addition is the same
for signed and unsigned binary numbers. Given a binary number
\texttt{\{$b^{31}$ $b^{30}$ ... $b^1$ $b^0$\}}.
If interpreted as an unsigned int, the value is
\begin{equation}
    b^{31} \times 2^{31} + b^{30} \times 2^{30} + \dots + b^{1} \times 2^{1} + b^{0} \times 2^{0}.
\end{equation}
The maximum is \texttt{11...11} ($2^{32} - 1$), and the minimum
is \texttt{00...00} (0).
If we instead interpret it as a signed integer, the value is
\begin{equation}
    -b^{31} \times 2^{31} + b^{30} \times 2^{30} + \dots + b^{1} \times 2^{1} + b^{0} \times 2^{0}.
\end{equation}
This means the maximum is \texttt{01...11} ($2^{31} - 1$),
and the minimum is \texttt{10...00} ($-2^{31}$).

To negate a 2's complement number, invert all bits and
add 1. To convert an $n$ bit number into a number with more than
$n$ bits, copy (sign extend) the MSB (the sign bit) into the other bits.
For example, \texttt{1001} becomes \texttt{11111001}.
\marginnote{Take a moment to justify to yourself why the negation
    and extension operations work.}

An ALU is shown in Figure \ref{fig:alu}.
\begin{figure}
    \includegraphics{images/alu.png}
    \caption{ALU}
    \label{fig:alu}
\end{figure}
Its structure reveals that for every number,
in fact the AND, OR, sum, and difference are
calculated, and a mux selects the results based
on the type of operation.

Overflow can occur when two positive numbers are
added, or two negative numbers are added. Overflow
can be detected with the XOR of the carry into MSB
and carry out of MSB.
\marginnote{No carry out is equivalent to a carry out of 0.}

Negative numbers and zero are required
for condition branches, e.g. \texttt{beq}
and \texttt{blt}.

\subsection{Integer Multiplication}

\subsection{Integer Division}

\subsection{Floating Point Arithmetic}