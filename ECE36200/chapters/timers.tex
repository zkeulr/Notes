\section{Timers}
In MCUs, we often want a way to periodically do something or
many somethings. Every useful MCU has a hardware timer system.
A \emph{system ticker} (systick) is a special timer reserved for OS
operations, but there is always a timer subsystem outside the CPU
for general use.

The systick timer is a piece of hardware inside the MCU that generates an
systick interrupt signal at fixed intervals. Every interrupt has a dedicated
ISR to handle it, and in the case of systick the ISR is \texttt{SysTick\_Handler}.

In the ARM Cortex-M, the system timer is built into the hardware of the
CPU. Every time an IRQ is raised, the Nested Vectored Interrupt
Controller (NVIC) hardware will determine whether or not to handle the
systick interrupt.

\begin{figure}
    \centering
    \begin{tikzpicture}[node distance=2.2cm and 2.2cm, font=\small]
        \tikzset{
        block/.style   = {draw, thick, rectangle, rounded corners=2pt, minimum width=3.2cm, minimum height=1.2cm, align=center},
        decision/.style= {draw, thick, diamond, aspect=2, inner sep=2pt, align=center},
        line/.style    = {thick, -{Latex[length=3mm]}}
        }

        % Counter and logic
        \node[block] (counter) {Counter\\(24-bit down)};
        \node[decision, right=of counter] (iszero) {count == 0?};
        \node[block, below=1.8cm of iszero] (reload) {Reload original\\value};

        % Connections
        \draw[line] (counter.east) -- (iszero.west);
        \draw[line] (iszero.east) -- ++(1.6,0) node[above, pos=0.45]{No} -- ++(0,1.4) -- ($ (counter.north) + (0,0.8) $) -- (counter.north);
        \draw[line] (iszero.south) -- node[right]{Yes} (reload.north);
        \draw[line] (reload.west) -| (counter.south);

        % Clock waveform feeding the counter
        \coordinate (clkstart) at ($(counter.west)+(-3.2,0)$);
        \node[left] at (clkstart) {Clock};
        \draw[thick]
        (clkstart)
        -- ++(0.6,0) -- ++(0,0.6) -- ++(0.6,0) -- ++(0,-0.6)
        -- ++(0.6,0) -- ++(0,0.6) -- ++(0.6,0);
        \draw[line] ($(clkstart)+(2.4,0)$) -- (counter.west);
    \end{tikzpicture}
    \caption{System Timer}
    \label{fig:systemtimerflowchart}
\end{figure}

When the counter hits zero, \texttt{COUNTFLAG} is set to 1.
The choice of clock input can vary, as most MCUs have multiple.
The clock input is ANDed with a flag to enable and disable.

Let's do an example. Suppose the clock tick frequency $f$ is
80MHz and the goal is a systick
interval $s$ of 10ms. What must the reload value $R$ be?
\begin{align}
    R & = sf - 1                \\
      & = 10ms \times 80MHz - 1 \\
      & = 800000 - 1            \\
      & = 799999
\end{align}

