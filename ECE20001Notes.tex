\documentclass[nobib]{tufte-handout}

%\\geometry{showframe}% for debugging purposes -- displays the margins

\newcommand{\bra}[1]{\left(#1\right)}
\usepackage{clrscode3e}
\usepackage{hyperref}
\usepackage[activate={true,nocompatibility},final,tracking=true,kerning=true,spacing=true,factor=1100,stretch=10,shrink=10]{microtype}
\usepackage{color}

\usepackage{tikz}
\usepackage{amsmath,amsthm}
\usetikzlibrary{shapes}
\usetikzlibrary{positioning}

% Set up the images/graphics package
\usepackage{graphicx}
\setkeys{Gin}{width=\linewidth,totalheight=\textheight,keepaspectratio}
\graphicspath{{.}}

\title{Notes for ECE 20001 - Electric Engineering Fundementals I}
\author[Ezekiel Ulrich]{Ezekiel Ulrich}
\date{\today}  % if the \date{} command is left out, the current date will be used

% The following package makes prettier tables.  We're all about the bling!
\usepackage{booktabs}

% The units package provides nice, non-stacked fractions and better spacing
% for units.
\usepackage{units}

% The fancyvrb package lets us customize the formatting of verbatim
% environments.  We use a slightly smaller font.
\usepackage{fancyvrb}
\fvset{fontsize=\normalsize}

% Small sections of multiple columns
\usepackage{multicol}

% These commands are used to pretty-print LaTeX commands
\newcommand{\doccmd}[1]{\texttt{\textbackslash#1}}% command name -- adds backslash automatically
\newcommand{\docopt}[1]{\ensuremath{\langle}\textrm{\textit{#1}}\ensuremath{\rangle}}% optional command argument
\newcommand{\docarg}[1]{\textrm{\textit{#1}}}% (required) command argument
\newenvironment{docspec}{\begin{quote}\noindent}{\end{quote}}% command specification environment
\newcommand{\docenv}[1]{\textsf{#1}}% environment name
\newcommand{\docpkg}[1]{\texttt{#1}}% package name
\newcommand{\doccls}[1]{\texttt{#1}}% document class name
\newcommand{\docclsopt}[1]{\texttt{#1}}% document class option name

% Define a custom command for definitions
\newcommand{\defn}[2]{\noindent\textbf{#1}:\ #2}

\begin{document}

\maketitle

\begin{abstract}
These are lecture notes for fall 2023 ECE 20001 at Purdue. Modify, use, and distribute as you please.
\end{abstract}

\tableofcontents

\section{Course Introduction}

This course covers fundamental concepts and applications 
for electrical and computer engineers as well as for engineers
 who need to gain a broad understanding of these disciplines. 
 The course starts by the basic concepts of charge, current, 
 and voltage as well as their expressions with regards to 
 resistors and resistive circuits. Essential concepts, 
 devices, theorems, and applications of direct-current (DC), 
 1st order, and alternating-current (AC) circuits are 
 subsequently discussed. Besides electrical devices and 
 circuits, basic electronic components including diodes and 
 transistors as well as their primary applications are also 
 discussed. For more information, see the syllabus. 

\section{Equations}

\begin{enumerate}
    \item $P = \frac{dW}{dt} = IV$
    \item $I = \frac{dq}{dt}$
    \item $V = \frac{W}{q}$
    \item $R = \frac{\rho L}{A}$
    \item Coulomb's Law: $\vec{F} = \frac{1}{4\pi \epsilon_0}\frac{q_1 q_2}{r^2}\hat{r}$
    \item Kirchhoff's Voltage Law: 
    \item Ohm's Law: $V=IR$
\end{enumerate}

\section{Charge, current, voltage, and power}

\defn{Charge}{A fundemental property of matter.}

\defn{Current}{The rate of flow of charge.}

\defn{Voltage}{Related to the potential energy of charges.}

\defn{Power}{The rate of doing work, or changing energy}

\defn{Passive sign convention}{Defines 
electric power flowing out of the circuit into an electrical 
component as positive, and power flowing into the circuit out 
of a component as negative. So a passive component which 
consumes power, such as an appliance or light bulb, will 
have positive power dissipation, while an active component, 
a source of power such as an electric generator or battery, 
will have negative power dissipation.}

It's useful to have an idea of the components of circuit
schematics (visual representations of a circuit). Below is a list 
of the terms that will be used in this course:
\begin{itemize}
    \item Elements: The term elements means "components and sources."
    \item Symbols: Elements are represented in schematics by symbols. 
    Symbols for common 2-terminal elements are displayed to the right.
    
\begin{marginfigure}
    \centering
    \includegraphics{images/symbols.png}
    \caption{Common circuit symbols}
    \label{fig:symbols}
\end{marginfigure} 

    \item Lines: Connections between elements are drawn as lines, 
    which we often think of as "wires". On a schematic, 
    these lines represent perfect conductors with zero resistance. 
    Every component or source terminal touched by a line is at the same voltage.
    \item Dots: Connections between lines can be indicated by dots. 
    Dots are an unambiguous indication that lines are connected. 
    If the connection is obvious, you don't have to use a dot.
\end{itemize}
Check out the circuit schematic below and see how many components you 
can identify!
\begin{center}
    \includegraphics[width=\textwidth/2]{images/PS0_CapacitorCircuit.png}
\end{center}

Now, on to what circuits are doing. For interesting things
to happen we need electrons flowing through those wires.
Current is how quickly
electrons are moving along, or in more formal terms the rate of
change of charge. That is, $I = \frac{dQ}{dt}$ 
Voltage (or electric potential) is the amount of work done in moving charged 
particles such as electrons between two points. Whenever we want to separate 
two oppositely (positive and negative) charged particles or push two similarly-charged
particles together, we need to do
work to overcome the force between them. The work done in 
separating them is stored as potential energy.
This stored energy is known as electric potential (or voltage) and is measured in volts.
Thus, $V = \frac{W}{q}$. Just as with potential energy, voltage is always 
measured as a difference between two points. 

\section{(In)dependent sources, connections, resistance and Ohm's Law}

\defn{Series Combination}{In a series combination, the elements 
are connected with end to end in contact, such that 
current flow is equal in all the elements in the 
combination}

\includegraphics[width=\textwidth/2]{images/Series_circuit.svg.png}

\defn{Parallel Combination}{When two or more resistances are 
connected between the same two points, they are said to be 
connected in parallel combination. In this case voltage is equal
across all elements}

\includegraphics[width=\textwidth/2]{images/220px-Resistors_in_parallel.svg.png}

Turning off a voltage source is equivalent to replacing it with a 
short circuit (line). Turning off a current source is equivalent to replacing it with an
open circuit (broken line). 

$Resistance = \rho * Length of resistor/ cross section area$ $\rho$ is the resistivity of the material 

Conductance = 1/R 

\end{document}
