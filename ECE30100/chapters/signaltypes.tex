\section{Classifying Signal Types}
Before we proceed we must be able to classify signal types.
There are five ways to divide signal types.
\begin{itemize}
    \item DT vs. CT
    \item Periodic vs. aperiodic
    \item Finite energy vs. finite power
    \item Even and odd
    \item Complex exponential
\end{itemize}

\subsection{DT vs. CT}
\begin{itemize}
    \item DT: $x[n]$ is a sequence of complex values, including purely real values.
          numbers. Example: $x[n] = \frac{n}{2}$.
    \item CT: $x(t)$ is complex (including purely real) and continuous for all real values of $t$.
          Example: $x(t) = \frac{t}{2} -jt$.
    \item Complex:
\end{itemize}

For DT, complex $x[n]$ can be represented in Cartesian or polar form.
For Cartesian,
\begin{equation}
    x[n] = x_{Re}[n] + jx_{Im}[n].
\end{equation}
For polar,
\begin{equation}
    x[n] = A[n]e^{j\Theta[n]}.
\end{equation}
We can swap between the two with Euler's formula.
\begin{align}
    A[n]e^{j\Theta[n]}     & = A[n]\cos(\Theta[n]) + jA[n]\sin(\Theta[n])                                        \\
    x_{Re}[n] + jx_{Im}[n] & = \sqrt{x_{Re}[n]^2 + x_{Im}[n]^2} \times e^{j\arctan(\frac{x_{Im}[n]}{x_{Re}[n]})}
\end{align}

\subsection{Energy vs. Power}
DT vs. CT is one option to classify signals. Another
possibility is Energy vs. Power. For this class, energy in a continuous
time system is the area under the squared magnitude of the signal. Mathematically
energy over $(t_1, t_2)$ is equal to
\begin{align}
    E & = \int_{t_1}^{t_2} |x(t)|^2 dt                     \\
      & = \int_{t_1}^{t_2} (x_{Re}(t)^2 + x_{Im}(t)^2) dt.
\end{align}
For DT systems, the formula for energy is
\begin{align}
    E & = \sum_{n=n_1}^{n_2} |x[n]|^2                     \\
      & = \sum_{n=n_1}^{n_2} (x_{Re}[n]^2 + x_{Im}[n]^2).
\end{align}
The total energy $E_\infty$ is the energy from $t = -\infty$ to $t = \infty$.

Power is energy per unit time, or in terms of calculus $P(t) = \frac{d}{dt}E(t)$.
For CT, average power is
\begin{equation}
    P_{avg} = \frac{1}{t_2 - t_1} \int_{t_1}^{t_2} |x(t)|^2 dt.
\end{equation}
For DT,
\begin{equation}
    P_{avg} = \frac{1}{n_2 - n_1 + 1} \sum_{n=n_1}^{n_2} |x[n]|^2.
\end{equation}
The overall average power is
\begin{equation}
    P_{\infty} = \lim_{T \rightarrow \infty} \frac{1}{2T} \int_{-T}^{T} |x(t)|^2 dt
\end{equation}
for CT and
\begin{equation}
    P_{\infty} = \lim_{N \rightarrow \infty} \frac{1}{2N + 1} \sum_{n=-N}^{N} |x[n]|^2
\end{equation}
for DT time.

\subsection{Complex exponential}
A CT complex exponential signal is of the form 
$x(t) = Ce^{\alpha t}$, where $C$ and $\alpha$ 
are in general complex. Alternatively, 
\begin{equation}
    x(t) = |C| e^{\sigma t} e^{j(\omega t + \phi)}
\end{equation}
where $\phi$ is the angle between the real axis and 
$C$ when plotted on the complex plane and $\alpha = \sigma + j\omega$. 

The $\sigma$ term determines whether the signal has exponential 
growth, decay, or neither. If $\sigma = 0$ then we are left 
with the periodic complex exponential $e^{j(\omega t + \phi)}$ 
with period $\frac{2\pi}{\omega}$. 

$\omega$ is called the fundemental frequency, 
$\phi$ is called the phase. 