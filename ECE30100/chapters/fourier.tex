\section{Fourier Transforms} \label{sec:fourier}

\subsection{Eigenfunctions and $H(s)$}

An \emph{eigenfunction} $x(t) = e^{j\omega t}$ is a function such that
$S\{x(t)\} = Ax(t)$ for LTI $S$. A property of functions of the form
$e^{j\omega t}$ is that convolution with $h(t)$ yields
\begin{equation}
    x(t) * h(t) = e^{j\omega t} \int_{-\infty}^{\infty} h(\tau) e^{-j\omega \tau} d\tau.
\end{equation}
This is also written as
\begin{equation}
    x(t) * h(t) = e^{j\omega t} \left( |H(j\omega)|e^{j \angle H(j\omega)} \right)
\end{equation}
$H$ is known as the transfer function as has the property that
$Y(S) = H(s)X(s)$, where $Y(s)$ and $X(s)$ are the Laplace transforms
of $y(t)$ and $x(t)$.

\section{Fourier Series}

Consider an arbitrary periodic CT signal $x(t)$ with fundemental
period $T$ and fundemental frequency $\omega_0$. The Fourier series
representation of $x(t)$ is
\begin{align}
    x(t) = \sum_{k=-\infty}^{\infty} a_k e^{jk\omega_0 t}                         \\
     & = a_0 + a_1e^{j\omega_0 t} + a_{-1}e^{-j\omega_0 t} + a_2e^{2j\omega_0 t},
\end{align}
where $a_k$ is the $k$th Fourier coefficient and can be found with the formula
\begin{equation}
    a_k = \frac{1}{T} \int^{<T>} x(t) e^{-jk\omega_0 t} dt
\end{equation}
with $<T>: [0, T], [-\frac{T}{2}, \frac{T}{2}, ...]$.

Notably, $a_0$ gives the DC component of the singal. In general the $a_{\pm k}$
are the $k$th harmonic components.