\section{System Properties}

\subsection{Memoryless}
A \emph{memoryless} system output is dependent
only on the input at time $t$, and not previous states.

\subsection{Invertable}
A system is \emph{invertable} is distinct
inputs lead to distinct outputs. A system is not
invertable if there exists any $x_1(t) \neq x_2(t)$ such
that $y_1(t) = y_2(t)$.

Three practical examples are modulation, sampling,
and encoding. I do not understand modulation and
cannot explain it. Encoding isn't always invertable,
e.g. lossy image compression.

For an invertable system $S_1$, there exist an inverse system
$S_2$ such that $S_2(S_1(x(t))) = S_1(S_2(x(t))) = x(t)$.

\subsection{Linear}

\subsection{Causality}
A system is \emph{causal} when the output at any time
is dependent only on past or present inputs. All memoryless
systems are trivially causal. An example of a non-causal
system is the moving average of a signal.

\subsection{Stability}
A system is stable if small changes in the input
lead only to small changes in the output. For instance,
if a pendulum is hanging down, it is stable to displacements.
But if it's a vertical pendulum, it is not stable.

Mathematically, bounded inputs produce bounded outputs.
For finite $B$, $M$,
\begin{equation}
    |x(t)| < B \rightarrow |y(t)| < M.
\end{equation}
