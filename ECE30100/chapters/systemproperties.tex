\section{System Properties}

\subsection{Memoryless}
A \emph{memoryless} system's output depends only on the input at time
\( t \) (or \( n \) for discrete-time systems), and not on past or
future states. For example, \( y(t) = 2x(t) \) is memoryless, while
\( y[n] = x[n-1] \) has memory.

Recall that for any system,
\begin{equation}
    y[n] = \sum_{k=-\infty}^{\infty} h[k]x[n-k]
\end{equation}
For a memoryless system, $y[n]$ must be dependent only on $x[n]$ and
not some past or future inputs. That forces $h[n]$ to be 0 for all values
of $n$ except 0. Thus, for a memoryless system, the impulse response
is the delta function times some scalar.

\subsection{Linear}
A system is \emph{linear} if it satisfies superposition and homogeneity. Formally, for any inputs \( x_1(t) \rightarrow y_1(t) \) and \( x_2(t) \rightarrow y_2(t) \), and constants \( a, b \), the system satisfies:
\[
    S\{a x_1(t) + b x_2(t)\} = a S\{x_1(t)\} + b S\{x_2(t)\} = a y_1(t) + b y_2(t).
\]
An example is \( y(t) = kx(t) \), where \( k \) is a constant.

\subsection{Time Invariance}
A system is \emph{time invariant} if a time shift in the input results in an identical shift in the output. For a discrete-time system \( S \), if:
\[
    x[n] \rightarrow y[n],
\]
then for any integer \( n_0 \):
\[
    x[n - n_0] \rightarrow y[n - n_0].
\]

Informally, a shift in the input signal must result in the output signal
being shifted the same amount.

For continuous-time systems, this becomes \( x(t - \tau) \rightarrow y(t - \tau) \). Time-dependent operations (e.g., \( y(t) = x(2t) \)) or explicit time-varying components make a system \emph{time variant}.

\subsection{Linear Time Invariant (LTI)}
An LTI system satisfies both linearity and time invariance.
LTI systems are fully characterized by their impulse response
\( h(t) \) or \( h[n] \), enabling analysis via convolution:
\( y(t) = x(t) * h(t) \). The impulse response is found by
plugging in $\delta(t)$ or $\delta[n]$ to the system. The result
is the impulse response. Mathematically,
\begin{align}
    h(t) & = S(\delta(t))  \\
    h[n] & = S(\delta[n]).
\end{align}


\subsection{Invertible}
A system is \emph{invertible} if distinct inputs produce distinct outputs. If there exist \( x_1(t) \neq x_2(t) \) such that \( y_1(t) = y_2(t) \), the system is not invertible. Examples include:
\begin{itemize}
    \item \textbf{Modulation}: \( y(t) = x(t)\cos(\omega t) \) (invertible with synchronous demodulation).
    \item \textbf{Sampling}: Invertible if the Nyquist criterion is satisfied.
    \item \textbf{Encoding}: Lossy compression (e.g., JPEG) is non-invertible.
\end{itemize}
For an invertible system \( S_1 \), there exists an inverse \( S_2 \) such that:
\[
    S_2(S_1(x(t))) = S_1(S_2(x(t))) = x(t).
\]

It would be useful to have a general method of determining the inverse,
so that we can algorithmically find the inverse or prove a system is
noninvertable. For LTI systems we know that $y[n] = x[n] * h[n]$. When
this is fed into the inverse system we get $x[n] = x[n] * h[n] * h_I[n]$.
So the impulse response of the inverse system convolved with the
impulse response of the original system must be $\delta$. We will find a
perfectly systematic way of determining $h_I[n]$ when we examine Fourier
transforms in the section on \nameref{sec:fourier}.


\subsection{Causal}
A system is \emph{causal} if its output at time \( t \) depends
only on present and past inputs. All memoryless systems are causal
(e.g., \( y(t) = x(t)^2 \)). A non-causal system (e.g., moving average
\( y[n] = \frac{1}{3}(x[n+1] + x[n] + x[n-1]) \)) requires future inputs.

For causal systems, the impulse response must be 0 for values of $n$ ($t$)
less than 0.

\subsection{Stable}
A system is \emph{stable} if bounded inputs produce
bounded outputs. Formally, there exist constants \( B, M > 0 \)
such that:
\[
    |x(t)| < B \ \forall t \implies |y(t)| < M \ \forall t.
\]
Examples:
\begin{itemize}
    \item Stable: \( y(t) = e^{-t}x(t) \).
    \item Unstable: \( y(t) = \int_{-\infty}^t x(\tau) d\tau \)
          (unbounded integral for a constant input).
\end{itemize}

For a stable LTI system, the impulse response must be absolutely summable.
That is,
\begin{equation}
    \sum_{k=-\infty}^{\infty} |h[k| < \infty.
\end{equation}
Or for continuous time, absolutely integrable
\begin{equation}
    \int_{-\infty}^{\infty} |h(\tau)| d\tau < \infty.
\end{equation}