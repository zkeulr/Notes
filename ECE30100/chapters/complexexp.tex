\section{Complex Exponential}
A CT complex exponential signal is of the form 
$x(t) = Ce^{\alpha t}$, where $C$ and $\alpha$ 
are in general complex. Alternatively, 
\begin{equation}
    x(t) = |C| e^{\sigma t} e^{j(\omega t + \phi)}
\end{equation}
where $\phi$ is the angle between the real axis and 
$C$ when plotted on the complex plane and $\alpha = \sigma + j\omega$. 

The $\sigma$ term determines whether the signal has exponential 
growth, decay, or neither. If $\sigma = 0$ then we are left 
with the periodic complex exponential $e^{j(\omega t + \phi)}$ 
with period $\frac{2\pi}{\omega}$. 

$\omega$ is called the fundemental frequency, 
$\phi$ is called the phase. 

The signal $x(t) = |C|e^{\sigma t}e^{j(\omega t + \phi)}$
forms a family of signals called harmonically 
related complex exponentials (HRCEs), each of the 
form 
\begin{equation}
    x_k(t) = e^{jk\omega_0 t}, k \in \mathbb{Z}.
\end{equation}
These signals will serve as our building 
blocks when we construct Fourier series 
of complex exponential signals later on. 

Let's now look at the discrete time 
complex exponential,
\begin{equation}
    x[n] = C \alpha^n.
\end{equation}
In general, $C$ and $\alpha$ can be complex. 
This can be rewritten 
\begin{eqnarray}
    x[n] = |C|e^{\sigma n} e^{j(\omega n + \phi)}. 
\end{eqnarray}
Where the continuous and discrete begin 
to diverge is when we consider the 
$e^{j(\omega n + \phi)}$ term. This 
term is not always periodic, unlike 
the case of continuous time. For 
the CT case, the fundemental 
frequency is $\omega$ and larger 
values of $\omega$ produce 
higher rates of oscillation. 

Now say we want to compute the 
fundemental period of $x[n] = \cos(3\pi n)$. 
In the continuous case it's easy, $\frac{2}{3}$.
In the discrete case, we need to 
find the least common multiple 
of the fundemental period 
of the CT signal (in this case, 
$\frac{2}{3}$) and the sampling 
period. This is why the 
discrete case may not be periodic.
Imagine the sampling period is 1
and the fundemental period of the 
CT signal is $2 \pi$. Then the 
least common multiple does not 
exist. 

Let's do an example problem. 
Consider the signal 
\begin{eqnarray}
    x(t) = e^{j2t} + e^{j5t}.
\end{eqnarray}
We can rewrite this, using 
the identity 
\begin{equation}
    \cos(\omega t) = \frac{e^{-j \omega t}+ e^{j\omega t}}{2}
\end{equation}
to the expression 
\begin{align}
    e^{j 3.5 t} (e^{j1.5t} + e^{-j 1.5 t}) \\
    &= 2 e^{j 3.5 t}\cos(1.5t) \\
    |x(t)| &= 2|\cos(1.5 t)|
\end{align}

\begin{align}
    x[n] &= C\alpha^n \\
    &= Ce^{\beta n}
\end{align}
is the general form of a discrete complex exponential and 
$\beta$ is in general complex. If $\alpha$ is real and less 
than 0, then $\beta$ must be complex. 