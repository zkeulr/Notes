\section{Modulation}
Signal modulation is the process of varying one or more properties of
a periodic waveform in electronics and telecommunication for the
purpose of transmitting information on a shared medium where signals might
otherwise interfere.
\subsection{Frequency Modulation}
Consider a generic signal $s(t)$ and its spectrum, $S(j\omega)$. Say
the spectrum is band-limited. That is, its is only nonzero between
$-\omega_1$ and $\omega_1$, with $s(0) = A$ being the max amplitude.
\begin{figure}
    \begin{tikzpicture}
        \begin{axis}[
                axis x line=middle,
                axis y line=middle,
                xlabel={$\omega$},
                ylabel={$|S(j\omega)|$},
                xtick={-2, 2},
                xticklabels={$-\omega_1$, $\omega_1$},
                ytick=\empty,
                ymin=0, ymax=1.2,
                xmin=-3, xmax=3,
                height=6cm,
                width=10cm,
                domain=-3:3
            ]
            \addplot[
                thick,
                domain=-3:-1,
            ] {1 - (x + 1)^2};

            \addplot[
                thick,
                domain=-1:1,
            ] {1};

            \addplot[
                thick,
                domain=1:3,
            ] {1 - (x - 1)^2};
        \end{axis}
    \end{tikzpicture}
    \caption{Spectrum of $s(t)$}
\end{figure}

The modulation property is thus:
\begin{equation}
    x(t)\cos(\omega_0 t) \overset{\mathcal{F}}{\leftrightarrow} \frac{1}{2}\left( X(j(\omega-\omega_0)) + X(j(\omega-\omega_0))\right)
\end{equation}
for CT and
\begin{equation}
    x[n]\cos(\omega_0 n)  \overset{\mathcal{F}}{\leftrightarrow}  \frac{1}{2}\left( X(e^{j(\omega-\omega_0)}) + X(e^{j(\omega-\omega_0)})\right)
\end{equation}
for DT.

\subsection{Amplitude Modulation}