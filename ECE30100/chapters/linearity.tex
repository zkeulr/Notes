\section{Linearity}
Readers are familiar with the concept
of linearity, which mathematically
may be expressed as
\begin{equation}
    f(a + b) = f(a) + f(b)
\end{equation}

Linear systems possess the property
of superposition, so given an
input as a sum of weighted inputs
the output is a sum of weighted outputs.

The necessary and sufficient
conditions for linearity in a CT
system are
if the input is $\alpha_1x_1(t)
    + \alpha_2x_2(t)$ the output
is $S(\alpha_1x_1(t)) + S(\alpha_2x_2(t))$.
Formally,
\begin{equation}
    S(\alpha_1x_1(t) + \alpha_2x_2(t)) = \alpha_1S(x_1(t)) + \alpha_2S(x_2(t)).
\end{equation}
Likewise for DT systems,
\begin{equation}
    S[\alpha_1x_1[t] + \alpha_2x_2[t]] = \alpha_1S[x_1[t]] + \alpha_2S[x_2[t]].
\end{equation}
This equality for hold for any real
valued $\alpha_1$ and $\alpha_2$.

Consider the CT system $S$ given
by $y(t) = tx(t)$. We are interested
in determining if the system is
linear. We test it with the definition
of linearity,
\begin{align}
    y(\alpha_1x_1(t) + \alpha_2x_2(t)) & = t(\alpha_1x_1(t) + \alpha_2x_2(t))    \\
                                       & = t\alpha_1x_1(t) + t\alpha_2x_2(t)     \\
                                       & = \alpha_1y(x_1(t)) + \alpha_2y(x_2(t))
\end{align}
Since this is the definition of linearity, the
system is linear.

Why do we care? We care because linearity
gives us many useful properties and
makes solving systems much easier.
If we know the output for any set of
inputs, we can find the output for
any linear combination of those inputs.
