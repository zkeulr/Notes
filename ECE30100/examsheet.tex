\documentclass[8pt]{article}

\usepackage{amsmath}
\usepackage{amssymb}
\usepackage{geometry}

\geometry{margin=0.2in}

\begin{document}

% Signal time transformations and plotting
% Signal periodicity and fundamental frequency/period
% Signal power and energy
% Even and odd decompositions of signals
% Complex exponential signals, and harmonically-related complex exponentials (HRCEs)
% System properties: Linearity, time-invariance, memory, causality, stability, invertibility
% What LTI means

For periodic functions \(x_1(t)\), \(x_2(t)\) with periods \(T_1\) and
\(T_2\) respectively, the period of \(x_1(t) + x_2(t)\) is \(LCM(T_1, T_2)\).

For periodic functions \(x_1[n]\), \(x_2[n]\) with periods \(N_1\) and
\(N_2\) respectively, the period of \(x_1[n] + x_2[n]\) is \(LCM(N_1, N_2)\).
\begin{align*}
    E & = \int_{t_1}^{t_2} |x(t)|^2 dt                     \\
      & = \int_{t_1}^{t_2} (x_{Re}(t)^2 + x_{Im}(t)^2) dt.
\end{align*}
\begin{align*}
    E & = \sum_{n=n_1}^{n_2} |x[n]|^2                     \\
      & = \sum_{n=n_1}^{n_2} (x_{Re}[n]^2 + x_{Im}[n]^2).
\end{align*}
\begin{equation*}
    P_{avg} = \frac{1}{t_2 - t_1} \int_{t_1}^{t_2} |x(t)|^2 dt.
\end{equation*}
\begin{equation*}
    P_{avg} = \frac{1}{n_2 - n_1 + 1} \sum_{n=n_1}^{n_2} |x[n]|^2.
\end{equation*}
\begin{equation*}
    P_{\infty} = \lim_{T \rightarrow \infty} \frac{1}{2T} \int_{-T}^{T} |x(t)|^2 dt
\end{equation*}
\begin{equation*}
    P_{\infty} = \lim_{N \rightarrow \infty} \frac{1}{2N + 1} \sum_{n=-N}^{N} |x[n]|^2
\end{equation*}
\begin{align*}
    x(t) & = x_{even}(t) + x_{odd}(t)                        \\
         & = \frac{x(t) + x(-t)}{2} + \frac{x(t) - x(-t)}{2} \\
\end{align*}
\begin{align*}
    x[n] & = x_{even}[n] + x_{odd}[n]                        \\
         & = \frac{x[n] + x[-n]}{2} + \frac{x[n] - x[-n]}{2} \\
\end{align*}
\begin{align*}
    x_k(t)   & = e^{jk\omega_0 t}, k \in \mathbb{Z} \\
    \omega_0 & = \frac{2\pi}{T}, \text{ period T}   \\
    x_k[n]   & = e^{jk\omega_0 n}, k \in \mathbb{Z} \\
    \omega_0 & = \frac{2\pi}{N}, \text{ period N}   \\
\end{align*}
\begin{itemize}
    \item A \emph{memoryless} system depends only on input at time
          \( t \) (or \( n \) for DT systems).
    \item A \emph{linear} system satisfies
          \(S\{a x_1(t) + b x_2(t)\} = a S\{x_1(t)\} + b S\{x_2(t)\}.\)
    \item A system is \emph{time invariant} if a time shift (\(t_0\) or \(n_0\)) in the input results in
          an identical shift in the output.
    \item A \emph{linear time invariant}'s impulse response \(h\) is found by calculating
          \(S(\delta(t))\) (or \(S(\delta[n])\)). \(y(t) = x(t) * h(t)\) (or \(y[n] = x[n] * h[n]\)).
    \item A system is \emph{invertible} if there exists
          an inverse \( S_2 \) such that
          \(S_2(S_1(x(t))) = S_1(S_2(x(t))) = x(t)\).
    \item A system is \emph{causal} if its output at time \( t \) (index \(n\)) depends
          only on present and past inputs.
    \item A system is \emph{stable} if there exist constants \( B, M > 0 \)
          such that
          \(|x(t)| < B \ \forall t \implies |y(t)| < M \ \forall t\).
\end{itemize}
\end{document}