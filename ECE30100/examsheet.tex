\documentclass[8pt]{article}

\usepackage{amsmath}
\usepackage{amssymb}
\usepackage{geometry}
\usepackage{array}
\usepackage{booktabs}

\geometry{margin=0.2in}

\begin{document}

% Signal time transformations and plotting
% Signal periodicity and fundamental frequency/period
% Signal power and energy
% Even and odd decompositions of signals
% Complex exponential signals, and harmonically-related complex exponentials (HRCEs)
% System properties: Linearity, time-invariance, memory, causality, stability, invertibility
% What LTI means

For periodic functions \(x_1(t)\), \(x_2(t)\) with periods \(T_1\) and
\(T_2\) respectively, the period of \(x_1(t) + x_2(t)\) is \(LCM(T_1, T_2)\).

For periodic functions \(x_1[n]\), \(x_2[n]\) with periods \(N_1\) and
\(N_2\) respectively, the period of \(x_1[n] + x_2[n]\) is \(LCM(N_1, N_2)\).
\begin{align*}
    E & = \int_{t_1}^{t_2} |x(t)|^2 dt                     \\
      & = \int_{t_1}^{t_2} (x_{Re}(t)^2 + x_{Im}(t)^2) dt.
\end{align*}
\begin{align*}
    E & = \sum_{n=n_1}^{n_2} |x[n]|^2                     \\
      & = \sum_{n=n_1}^{n_2} (x_{Re}[n]^2 + x_{Im}[n]^2).
\end{align*}
\begin{equation*}
    P_{avg} = \frac{1}{t_2 - t_1} \int_{t_1}^{t_2} |x(t)|^2 dt.
\end{equation*}
\begin{equation*}
    P_{avg} = \frac{1}{n_2 - n_1 + 1} \sum_{n=n_1}^{n_2} |x[n]|^2.
\end{equation*}
\begin{equation*}
    P_{\infty} = \lim_{T \rightarrow \infty} \frac{1}{2T} \int_{-T}^{T} |x(t)|^2 dt
\end{equation*}
\begin{equation*}
    P_{\infty} = \lim_{N \rightarrow \infty} \frac{1}{2N + 1} \sum_{n=-N}^{N} |x[n]|^2
\end{equation*}
\begin{align*}
    x(t) & = x_{even}(t) + x_{odd}(t)                        \\
         & = \frac{x(t) + x(-t)}{2} + \frac{x(t) - x(-t)}{2} \\
\end{align*}
\begin{align*}
    x[n] & = x_{even}[n] + x_{odd}[n]                        \\
         & = \frac{x[n] + x[-n]}{2} + \frac{x[n] - x[-n]}{2} \\
\end{align*}
\begin{align*}
    x_k(t)   & = e^{jk\omega_0 t}, k \in \mathbb{Z} \\
    \omega_0 & = \frac{2\pi}{T}, \text{ period T}   \\
    x_k[n]   & = e^{jk\omega_0 n}, k \in \mathbb{Z} \\
    \omega_0 & = \frac{2\pi}{N}, \text{ period N}   \\
\end{align*}
\begin{itemize}
    \item A \emph{memoryless} system depends only on input at time
          \( t \) (or \( n \) for DT systems).
    \item A \emph{linear} system satisfies
          \(S\{a x_1(t) + b x_2(t)\} = a S\{x_1(t)\} + b S\{x_2(t)\}.\)
    \item A system is \emph{time invariant} if a time shift (\(t_0\) or \(n_0\)) in the input results in
          an identical shift in the output.
    \item A \emph{linear time invariant}'s impulse response \(h\) is found by calculating
          \(S(\delta(t))\) (or \(S(\delta[n])\)). \(y(t) = x(t) * h(t)\) (or \(y[n] = x[n] * h[n]\)).
    \item A system is \emph{invertible} if there exists
          an inverse \( S_2 \) such that
          \(S_2(S_1(x(t))) = S_1(S_2(x(t))) = x(t)\).
    \item A system is \emph{causal} if its output at time \( t \) (index \(n\)) depends
          only on present and past inputs.
    \item A system is \emph{stable} if there exist constants \( B, M > 0 \)
          such that
          \(|x(t)| < B \ \forall t \implies |y(t)| < M \ \forall t\).
\end{itemize}

\begin{table}[ht]
    \centering
    \label{tab:ctfourier_properties}
    \begin{tabular}{llll}
        \textbf{Property}    & \textbf{Time Domain}             &                                 & \textbf{Frequency Domain}                 \\
        Linearity            & $A x_1(t) + B x_2(t)$            & $\overset{FS}{\leftrightarrow}$ & $A a_k + B b_k$                           \\
        Even Symmetry        & $x(t)$ even                      & $\overset{FS}{\leftrightarrow}$ & $a_k$ even                                \\
        Odd Symmetry         & $x(t)$ odd                       & $\overset{FS}{\leftrightarrow}$ & $a_k$ odd                                 \\
        Time Shifting        & $x(t - t_0)$                     & $\overset{FS}{\leftrightarrow}$ & $a_k e^{-j k \omega_0 t_0}$               \\
        Frequency Shifting   & $x(t) e^{j n \omega_0 t}$        & $\overset{FS}{\leftrightarrow}$ & $a_{k - n}$                               \\
        Time Reversal        & $x(-t)$                          & $\overset{FS}{\leftrightarrow}$ & $a_{-k}$                                  \\
        Conjugation          & $x^*(t)$                         & $\overset{FS}{\leftrightarrow}$ & $a_{-k}^*$                                \\
        Periodic Convolution & $(x \ast y)(t)$                  & $\overset{FS}{\leftrightarrow}$ & $a_k b_k$                                 \\
        Multiplication       & $x(t) y(t)$                      & $\overset{FS}{\leftrightarrow}$ & $\sum_{n=-\infty}^{\infty} a_n b_{k - n}$ \\
        Differentiation      & $\frac{d}{dt} x(t)$              & $\overset{FS}{\leftrightarrow}$ & $j k \omega_0 a_k$                        \\
        Integration          & $\int x(t) dt$                   & $\overset{FS}{\leftrightarrow}$ & $\frac{a_k}{j k \omega_0}$ if $a_0 = 0$)  \\
        Parseval's Theorem   & $\frac{1}{T} \int_T |x(t)|^2 dt$ & $\overset{FS}{\leftrightarrow}$ & $\sum_{k=-\infty}^{\infty} |a_k|^2$       \\
    \end{tabular}
\end{table}

\begin{table}[ht]
    \centering
    6    \label{tab:dtfourier_properties}
    \begin{tabular}{llll}
        \textbf{Property}    & \textbf{Time Domain}                    &                                     & \textbf{Frequency Domain}                                       \\
        Linearity            & \(A\,x_1[n] + B\,x_2[n]\)               & \(\overset{DTFS}{\leftrightarrow}\) & \(A\,a_k + B\,b_k\)                                             \\
        Even Symmetry        & \(x[n]\) even (i.e., \(x[n]=x[-n]\))    & \(\overset{DTFS}{\leftrightarrow}\) & \(a_k\) even                                                    \\
        Odd Symmetry         & \(x[n]\) odd (i.e., \(x[n]=-x[-n]\))    & \(\overset{DTFS}{\leftrightarrow}\) & \(a_k\) odd                                                     \\
        Time Shifting        & \(x[n-n_0]\)                            & \(\overset{DTFS}{\leftrightarrow}\) & \(a_k\,e^{-j\frac{2\pi}{N} k n_0}\)                             \\
        Frequency Shifting   & \(x[n]\,e^{j\frac{2\pi}{N} n_0 n}\)     & \(\overset{DTFS}{\leftrightarrow}\) & \(a_{(k-n_0)\,\mathrm{mod}\,N}\)                                \\
        Time Reversal        & \(x[-n]\)                               & \(\overset{DTFS}{\leftrightarrow}\) & \(a_{-k}\) (indices mod \(N\))                                  \\
        Conjugation          & \(x^*[n]\)                              & \(\overset{DTFS}{\leftrightarrow}\) & \(a^*_{-k}\)                                                    \\
        Circular Convolution & \((x\circledast y)[n]\)                 & \(\overset{DTFS}{\leftrightarrow}\) & \(a_k\,b_k\)                                                    \\
        Multiplication       & \(x[n]\,y[n]\)                          & \(\overset{DTFS}{\leftrightarrow}\) & \(\frac{1}{N}\sum_{m=0}^{N-1} a_m\,b_{(k-m)\,\mathrm{mod}\,N}\) \\
        Difference Operator  & \(x[n]-x[n-1]\)                         & \(\overset{DTFS}{\leftrightarrow}\) & \(a_k\Bigl(1-e^{-j\frac{2\pi}{N} k}\Bigr)\)                     \\
        Parseval's Theorem   & \(\frac{1}{N}\sum_{n=0}^{N-1}|x[n]|^2\) & \(\overset{DTFS}{\leftrightarrow}\) & \(\sum_{k=0}^{N-1}|a_k|^2\)                                     \\
    \end{tabular}
\end{table}

\begin{itemize}
    \item $x(t)$ real $\rightarrow$ $a_k = a^*_{-k}$
    \item For a memoryless system, the impulse response
          is the delta function times a scalar.
    \item For causal systems, the impulse response must be 0 for values of $n$ ($t$)
          less than 0.
    \item For a stable LTI system, the impulse response must be absolutely summable.
          That is,
          \begin{equation}
              \sum_{k=-\infty}^{\infty} |h[k]| < \infty.
          \end{equation}
          Or for continuous time, absolutely integrable
          \begin{equation}
              \int_{-\infty}^{\infty} |h(\tau)| d\tau < \infty.
          \end{equation}
\end{itemize}

\begin{equation}
    x(t) = \sum_{k=-\infty}^{\infty} a_k e^{jk\omega_0 t}
\end{equation}

\begin{equation}
    a_k = \frac{1}{T} \int_{<T>} x(t) e^{-jk\omega_0 t} dt
\end{equation}

\begin{equation}
    x[n] = \sum_{k=<N>} a_k x_k[n]
\end{equation}

\begin{equation}
    a_k = \frac{1}{N} \sum_{n=<N>} x[n] e^{-jk\frac{2\pi}{N}n},
\end{equation}

\begin{align}
    x(t) * h(t) & = \int_{-\infty}^{\infty} x(\tau) h(t - \tau) d\tau \\
    x[n] * h[n] & = \sum_{k=-\infty}^{\infty} x[k] h[n - k].
\end{align}

\end{document}