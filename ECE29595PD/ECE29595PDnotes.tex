\documentclass[nobib]{tufte-handout}

%\\geometry{showframe}% for debugging purposes -- displays the margins

\newcommand{\bra}[1]{\left(#1\right)}
\usepackage{amssymb}
\usepackage{hyperref}
\usepackage[activate={true,nocompatibility},final,tracking=true,kerning=true,spacing=true,factor=1100,stretch=10,shrink=10]{microtype}
\usepackage{color}

% Fixes captions and images being cut off
\usepackage{marginfix}
\usepackage{array}
\usepackage{tikz}
\usepackage{amsmath,amsthm}
\usetikzlibrary{shapes}
\usetikzlibrary{positioning}
\usepackage{listings}
\usepackage{caption}
\DeclareCaptionFont{white}{\color{white}}
\DeclareCaptionFormat{listing}{\colorbox{gray}{\parbox{\textwidth}{#1#2#3}}}
\captionsetup[lstlisting]{format=listing,labelfont=white,textfont=white}

% Set up the images/graphics package
\usepackage{graphicx}
\setkeys{Gin}{width=\linewidth,totalheight=\textheight,keepaspectratio}
\graphicspath{{.}}

\title{Notes for ECE 29595PD - Principles of Digital System Design}
\author[Shubham Saluja Kumar Agarwal]{Shubham Saluja Kumar Agarwal}
\date{\today}  % if the \date{} command is left out, the current date will be used

% The following package makes prettier tables.  We're all about the bling!
\usepackage{booktabs}

% The units package provides nice, non-stacked fractions and better spacing
% for units.
\usepackage{units}

% The fancyvrb package lets us customize the formatting of verbatim
% environments.  We use a slightly smaller font.
\usepackage{fancyvrb}
\fvset{fontsize=\normalsize}

% Small sections of multiple columns
\usepackage{multicol}

% For finite state machines 
\usetikzlibrary{automata} % Import library for drawing automata
\usetikzlibrary{positioning} % ...positioning nodes
\usetikzlibrary{arrows} % ...customizing arrows
\tikzset{node distance=2.5cm, % Minimum distance between two nodes. Change if necessary.
    every state/.style={ % Sets the properties for each state
    semithick,
    fill=gray!10},
    initial text={}, % No label on start arrow
    double distance=2pt, % Adjust appearance of accept states
    every edge/.style={ % Sets the properties for each transition
    draw,
    ->,>=stealth', % Makes edges directed with bold arrowheads
    auto,
    semithick}}
\let\epsilon\varepsilon

% These commands are used to pretty-print LaTeX commands
\newcommand{\doccmd}[1]{\texttt{\textbackslash#1}}% command name -- adds backslash automatically
\newcommand{\docopt}[1]{\ensuremath{\langle}\textrm{\textit{#1}}\ensuremath{\rangle}}% optional command argument
\newcommand{\docarg}[1]{\textrm{\textit{#1}}}% (required) command argument
\newenvironment{docspec}{\begin{quote}\noindent}{\end{quote}}% command specification environment
\newcommand{\docenv}[1]{\textsf{#1}}% environment name
\newcommand{\docpkg}[1]{\texttt{#1}}% package name
\newcommand{\doccls}[1]{\texttt{#1}}% document class name
\newcommand{\docclsopt}[1]{\texttt{#1}}% document class option name

% Define a custom command for definitions and biconditional
\newcommand{\defn}[2]{\noindent\textbf{#1}:\ #2}
\let\biconditional\leftrightarrow

\begin{document}

\maketitle

\begin{abstract}
These are lecture notes for spring 2024 ECE 29595PD at Purdue. Modify, use, and distribute as you please.
\end{abstract}

\tableofcontents

\section{Course Introduction}

This course serves as an introduction to digital system design, with an emphasis
on principles of digital hardware and embedded system design. It is an alternate class to ECE 27000. \\
Learning Outcomes:
\begin{enumerate}
    \item Ability to analyze and design combinational logic circuits.
    \item Ability to analyze and design sequential logic circuits.
    \item Ability to analyze and design computer logic circuits.
    \item Ability to realize, test, and debug practical digital circuits.
\end{enumerate}

\pagebreak 

\section{Introduction}

Digital design entails creating hardware that can conduct an operation or set of operations within a computer system. For example, adding two numbers. \\
\begin{center}
    \includegraphics[width= 100px]{images/Screenshot 2024-01-08 151414.png}
\end{center}
This hardware can add two numbers, that is, conduct the operation $c=a+b$. It could also perform $f=d+e$ or $i=g-h=g+(-h)$. This process fits into the logic design and switching algebra portions of chip manufacturing.\\
All of this is based on the fact that voltage and current are time-varying and can assume any value in a continuous range of real numbers, but are mapped to only two values.

\subsection{Digital Logic Signals}
A digital signal is modeled as assuming, at anytime, only one of two discrete values, called:\\
\begin{table}
 \centering
    \begin{tabular}{c|c}
    0 & 1 \\
    LOW & HIGH \\
    TRUE & FALSE\\
    \end{tabular}
\end{table}
This is called positive logic.
This maps the infinite values of voltage and current to the two values.
An example of this is CMOS 2-Volt logic:\\
\begin{table}
 \centering
    \begin{tabular}{c|c}
    0 & 1 \\
    \hline
    0-0.5V & 1.5-2.0V \\
    \end{tabular}
\end{table}
These completely separated ranges of values allow for 0 and 1 to be completely separate, with noise and other possible errors being ignored.
\pagebreak
\end{document}