\section{Computer Networks}

The high-level question this course will answer is "how do computers reliably
communicate?"

The answer is through computer networks, a group
of interconnected nodes or computing devices that
exchange data and resources with each other.

\begin{figure}[htbp]
    \centering
    \begin{tikzpicture}[
            node distance=2cm,
            router/.style={rectangle, draw, fill=blue!20, minimum width=1cm, minimum height=0.6cm},
            switch/.style={rectangle, draw, fill=green!20, minimum width=1cm, minimum height=0.6cm},
            computer/.style={circle, draw, fill=yellow!20, minimum size=0.8cm},
            server/.style={rectangle, draw, fill=red!20, minimum width=1cm, minimum height=1cm},
            internet/.style={cloud, draw, fill=gray!20, minimum width=2cm, minimum height=1cm}
        ]

        % Internet cloud
        \node[internet] (internet) at (0,4) {Internet};

        % Core router
        \node[router] (core) at (0,2) {R1};

        % Distribution layer
        \node[router] (r2) at (-3,0) {R2};
        \node[router] (r3) at (3,0) {R3};

        % Access layer switches
        \node[switch] (s1) at (-4.5,-2) {S1};
        \node[switch] (s2) at (-1.5,-2) {S2};
        \node[switch] (s3) at (1.5,-2) {S3};
        \node[switch] (s4) at (4.5,-2) {S4};

        % End devices
        \node[computer] (pc1) at (-5.5,-4) {PC1};
        \node[computer] (pc2) at (-3.5,-4) {PC2};
        \node[server] (srv1) at (-2,-4) {Web};
        \node[server] (srv2) at (-1,-4) {DB};
        \node[computer] (pc3) at (1,-4) {PC3};
        \node[computer] (pc4) at (2.5,-4) {PC4};
        \node[server] (srv3) at (4,-4) {File};
        \node[computer] (pc5) at (5.5,-4) {PC5};

        % Connections
        % Internet to core
        \draw[thick] (internet) -- (core);

        % Core to distribution
        \draw[thick] (core) -- (r2);
        \draw[thick] (core) -- (r3);

        % Distribution to access
        \draw[thick] (r2) -- (s1);
        \draw[thick] (r2) -- (s2);
        \draw[thick] (r3) -- (s3);
        \draw[thick] (r3) -- (s4);

        % Redundant connections between distribution routers
        \draw[thick, dashed] (r2) -- (r3);

        % Access to end devices
        \draw (s1) -- (pc1);
        \draw (s1) -- (pc2);
        \draw (s2) -- (srv1);
        \draw (s2) -- (srv2);
        \draw (s3) -- (pc3);
        \draw (s3) -- (pc4);
        \draw (s4) -- (srv3);
        \draw (s4) -- (pc5);

        % Some cross-connections for redundancy
        \draw[dotted] (s1) -- (s2);
        \draw[dotted] (s3) -- (s4);

    \end{tikzpicture}
    \caption{Computer Network}
    \label{fig:computer-network}
\end{figure}

A computer network enables communication between users and their devices.

\begin{figure}[htbp]
    \centering
    \begin{tikzpicture}[
            node distance=2.5cm,
            user/.style={ellipse, draw, fill=blue!20, minimum width=1.5cm, minimum height=0.8cm},
            device/.style={rectangle, draw, fill=yellow!20, minimum width=1cm, minimum height=0.6cm},
            router/.style={diamond, draw, fill=green!20, minimum width=1cm, minimum height=1cm},
            switch/.style={regular polygon, regular polygon sides=6, draw, fill=orange!20, minimum size=0.8cm}
        ]

        \node[user] (alice) at (-4,0) {Alice's PC};

        \node[user] (bob) at (4,0) {Bob's tablet};

        % Intermediate nodes
        \node[router] (r1) at (-1,0) {R1};
        \node[switch] (s1) at (0,1) {S1};
        \node[router] (r2) at (1,0) {R2};

        % Network path between Alice and Bob
        \draw[thick, blue] (alice) -- (r1);
        \draw[thick, blue] (r1) -- (s1);
        \draw[thick, blue] (s1) -- (r2);
        \draw[thick, blue] (r2) -- (bob);

        % Alternative path (dashed)
        \draw[thick, dashed, red] (r1) -- (r2);

    \end{tikzpicture}
    \caption{Simple Network}
    \label{fig:simple-network}
\end{figure}

The first stop that most home devices take in connecting to other
devices is the router. From there, the router can route data and
find a path for Alice's PC to get to Bob's tablet. The core of any
network is routers, which figure out the best way to get data from
one device to another device.

This process can get complex, with cell tower connections, different ISPs,
different edge devices, and more. Let's abstract the important elements
of the network.

\begin{itemize}
    \item Links: carry data from one endpoint to another
    \item End hosts: sitting at the edge of a network. Generate and receive data.
    \item Routers: forward data through the network.
\end{itemize}

Any network can be abstracted as a connection of links, end hosts, and routers.
We can thus represent any computer network as a graph, in the mathematical
sense, and apply all our graph algorithms to it.