\section{Packets}
In our abstraction of routers, we leave the problem of finding
a path from end hosts unanswered. Since we can represent a computer
network as a graph, a shortest-path algorithm like Dijkstra's is a
relatively simple way to find a good path between Alice and Bob.

We wish to answer the question "how do users access shared network
resources?" Assume no coordination between users and assume users
initiate access.

There are two ways to answer this:
\begin{itemize}
    \item Reserving resources (circuit switching).
    \item On-demand (packet switching)
\end{itemize}

\subsection{Circuit Switching}
The idea at the core of this method is that if Alice wants to communicate
with Bob, then they reserve a path through the network to do that. The
reserved path is called a \emph{circuit}. They then send data along the
reserved path.

A pro of circuit switching is that users are guaranteed to have a path
through the network. No queuing, no waiting. The routers along the path
also don't need to make any decisions in real time. They know the path
in advance.

A con is that a resource could be reserved but not used. When users don't
coordinate, circuit switching leads to very inefficient use of resources.
Another con is that circuits need to be set up and torn down. Imagine the
overhead of old telephone networks, where workers had to manually connect
telephone wires from one caller to another.

In classical circuit switching, all the resources along the path are reserved
for a single circuit and there is no path sharing amongst multiple circuits.
In virtual circuit switching, each circuit reserves a subset of resources
along its path. It's still reservation based, but two or more circuits can share
a same resource (e.g. if two users want to send data that takes up half the
bandwidth of a connection, they can both send it at once).

\subsection{Packet Switching}
To overcome the shortcomings of circuit switching, packet switching was invented.
The way this works is data is broken into small units called \emph{packets}.
You send packets over the network whenever you have them and trust the routers
to figure out the path your packets take on the fly.

Each packet needs to have metadata that describes things like the destination
of the packets, the order of the packets, and so on.

A pro of packet switching is that we have much better resource utilization.
Also, there's no overhead of circuit setup and teardown.

A con of packet switching is that there's no guarantee that a given user
will have the resources to send their data. Now we also have overhead from
the packet headers, and routers now need to process packet headers and find paths
on the fly.

Although both have advantages and disadvantages, modern networks almost
exclusively use packet switching and unless otherwise specified all
networks in this class are assumed to be packet switched.

\subsection{Packet Headers}
Packet headers require, at least, the following:
\begin{itemize}
    \item Destination address, used by network to send packet to destination
    \item Destination port, used by network stack at the destination for
          application multiplexing
    \item Source address, used by network to send packet back to source
    \item Source port, used by network stack at the source for application
          multiplexing
\end{itemize}

We'll learn more about the OSI model in a later section, but for now suffice
to say that each layer has its own header that it stacks onto and strips
from user data, starting with the application layer attaching its own
header to user data, to the transport layer sticking a header on, and so on.
As the package travels from host to destination the layers get added, and
then stripped off in the order they were added.