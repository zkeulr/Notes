\subsection{Border Gateway Protocol}

Recall that the internet is a hierarchical network of networks.
An autonomous system (AS) or domain is a network under a single
administrative domain. Each AS is assigned a unique identifier.

Intra-domain routing refers to routing within an AS. Examples:
RIP (distance vector), OSPF (link state).

Inter-domain routing refers to routing between ASes. Example: border gateway
protocol (BGP).

We can't simply use DV or LS for inter-domain routing because ASes want freedom
to pick routes based on custom policy. This can't be expressed as a least cost path
like in DV or LS, but rather depends on the business relationship between
ASes.

There are three basic kinds of relationships between ASes.

\begin{itemize}
      \item Customer, e.g. Purdue is AT\&T's customer. The customer pays the provider.
      \item Provider, e.g. AT\&T is Purdue's provider.
      \item Peer, e.g. AT\&T and Verizon are peers. Peers don't pay each other.
            Peers exchange roughly equal levels of traffic.
\end{itemize}

A customer is multi-homing if it has multiple providers. Provider ASes connect
customer ASes to the rest of the Internet, and customer ASes
pay the provider ASes for this service. Peering is needed to connect ASes
at the top, which have no providers. These ASes are known as Tier 1 ASes and
include ISPs like AT\&T, Sprint, Verizon. Peering can also be done at lower
levels for various reasons, such as two top secret institutions not trusting
a provider and exchanging information directly.

To set up inter-domain routing, destinations are subnets. Nodes are ASes, and
the internals of each AS are hidden. Physical links between ASes are tagged
with the corresponding business relationship. Border gateway protocol (BGP) is
user for routing.

In BGP, each AS advertises to neighbor ASes its best paths to one or more
subnets. Each AS selects the "best" path(s) from the set of advertised paths
to a subnet destination.

BGP is inspired by distance vector. It has per-distance route advertisements to
its neighbors. There is no global learning of network topology. It's iterative
and uses distributed convergence to final paths. But, there are three key differences.

\begin{itemize}
      \item While distance vector shares the distance and next hop to a destination,
            BGP shares the entire path to a destination to enable ASes to apply more
            complex policies. This makes loop detection trivial.
      \item BGP does not always select the shortest path.
      \item For policy reasons, an AS may choose not to advertise a route. Thus
            reliability is not guaranteed even if the graph is connected.
\end{itemize}

BGP policy is imposed in how route are selected and advertised. Selection
is which path to use to send data. It controls whether/how traffic leaves
the network. Advertisement is which path to advertise to other ASes. It
controls whether/how traffic enter the network. Traffic flows in the reverse
direction of the advertisement.

Typical selection policies are, in decreasing order of priority,
\begin{enumerate}
      \item Make or save money by preferring a path through the customer,
            then a path through a peer, then a path through a provider
      \item Maximize performance by selecting the shortest path
      \item Minimize use of network bandwidth aka "hot potato" routing
\end{enumerate}

The typical advertisement policy is Gao-Rexford advertisement policies,
whose goal is to avoid being transit when there is no monetary gain.
The customer advertises to providers, peers, and other customers. The
peer advertises to customers. The provider advertises to costumers.
The AS provides transit service to all its customer traffic, because that's
what customers pay for. It does not provide transit service between two
providers or peers, since these entities are not paying the AS.

eBGP refers to  BGP sessions between gateway (border) routers in
different ASes, used to learn routes to external destinations.
iBGP refers to BGP sessions between gateway routers and other
routers (both gateway and interior) within the same AS, used to
distribute externally learned routes internally.

IGP, Interior Gateway Protocol, is an intra-domain routing protocol that
provides internal connectivity.

Basic messages in BGP are
\begin{itemize}
      \item Open message: establishes BGP session, relayed on a reliable transport layer like TCP.
      \item Update message: Advertises new routes or route changes to neighbors and updates
            neighbors of any old routes that have become inactive.
      \item Keep-alive message: informs neighbor that BGP session is still active.
\end{itemize}

Update messages have the format \{Destination IP prefix: Update type | Route attributes\}.
There are two update types: announcement, which heralds new route or changes to
existing routes, and withdrawal, which announces the removal of routes which no longer
exist. Route attributes are used in route selection and advertisement choices. They
include \texttt{ASPATH}, a vector of ASes a BGP message has traversed, and
\texttt{LOCAL PREF}, preference value for \texttt{ASPATH}, among others.

In BGP, it's very easy to manipulate route advertisements. Reachability is
not guaranteed even if a graph is connected. More seriously, there are
attacks which an AS can perform by manipulating advertisements. An AS can
advertise an IP prefix that they actually don't have a route to, or
advertise a more specific IP prefix which, due to longest prefix match,
all traffic to the sub-prefixes will be redirected. Thus ASes can act as
black holes causing all traffic to the hijacked prefix to be discarded,
they can snoop by inspecting traffic to the hijacked prefix, and redirect
traffic to a hijacked prefix to bogus destinations.
BGP path advertisements can also be manipulated to show shorter paths
or more connected ASes. These attacks happen relatively infrequently
since one needs access to BGP routers to launch most attacks and
it's easy to detect the culprit. Most BGP mishaps are the result
of unintentional misconfiguration.