\section{OSI Model}

The Open Systems Interconnection (OSI) model describes
communications from the physical implementation of
transmitting bits across a transmission medium to the
highest-level representation of data of a distributed
application. Each layer has well-defined functions and
semantics and serves a class of functionality to the
layer above it and is served by the layer below it.

\subsection{Physical}

The realm of electrical and computer engineers.
Deals with converting between digital and analog signals or
electrical and optical signals. Beyond the purview of
this course.

\subsection{Data Link}
The data link layer runs on top of the physical layer.
It transfers data between nodes on a network segment
across the physical layer. Whereas the internet as a whole
runs on a global standard (IP) to allow subnetworks to
communicate, the data link layer allows autonomy
within each local area network (LAN).
Each LAN can run its own network
protocol for communication within LAN,
e.g., Ethernet, Wi-Fi, 5G, CSMA, Sonet, etc.
The data link layers handles addressing,
destination discovery, forwarding, and routing within
a local network. We will study data link layer mechanisms in the
context of the most popular data link layer protocol
called “Ethernet”
Ethernet is an example of a wired data link layer protocol,
i.e., nodes are connected using physical cables
\marginnote{
    Another very popular data link layer protocol is Wi-Fi,
    which is an example of a wireless data link layer protocol.
    Wi-Fi is not covered in this class}

\paragraph{MAC Addresses}

All network devices are connected to the network via a
“Network Interface” or “Port”. A network interface can be
“physical” (wired or wireless), such as an actual connection
on a server in some closet, or it can be “virtual”, i.e., a
piece of software emulating a network interface.
Each network interface, physical or virtual, has a Media Access
Control (MAC) address. MAC addresses are 48 bits or 6 bytes long and
typically represented in hexadecimal format, e.g., \texttt{ab:00:05:2c:e4:34}.
MAC address of each interface within a given network must be unique,
but MAC addresses are not necessarily globally unique.

There are three ways to transmit information from a sender to a
recipient:
\begin{itemize}
    \item Unicast: one-to-one transmission
    \item Multicast: many-to-many transmission
    \item Broadcast: one-to-all transmission
\end{itemize}

Naive implementations of
broadcast might have the sender send its packet to every of the
$N-1$ hosts in the network, but a more efficient implementation
is to send the packet to the router and have it send it out
to everyone else.
A special destination MAC address of \texttt{ff:ff:ff:ff:ff:ff} is used
to indicate a broadcast packet.

To see the list of interfaces on your computer, run the following
command in the terminal: \texttt{ifconfig} (mac/Linux) or
\texttt{ipconfig} (Windows).

In Ethernet, the Ethernet data (payload) and header is carried in an “Ethernet Frame”.
The structure of an Ethernet frame is as follows:
All Ethernet packets start with a “Preamble” - 7 bytes of alternating 1s and 0s
used for clock synchronization between sender and receiver.
This is followed by the Start Frame Delimiter (SFD), the one byte
\texttt{10101011}, then the destination MAX address, 6 bytes, and
then the source MAC address, 6 bytes. These are followed by the
Ethernet type, 2 bytes, which specifies the protocol carried in
the payload of the packet (e.g. IP), and finally the data itself.
The data has a minimum size of 46 bytes and a maximum size specified
by the Maximum Transmission Unit (MTU), which is configurable.
Everything is capped off with a Frame Check Sequence (FCS) of
four bytes, used for bit error correction and detection, and an
Inter Packet Gap (IPG), which is minimum 12 bytes of all 0s.

\paragraph{ARP}

The Address Resolution Protocol (ARP) is used to get the MAC
address of a destination host
within the same local network as the source host. It
assumes you know the IP address of the destination host.
Each host maintains a local table called ARP table
which stores a mapping between an IP address and MAC address.
Run \texttt{arp -a} on mac/Linux to view the table,
If the entry is found in the table, done!
Else run ARP to get the MAC address.

The ARP protocol has three stages. Say a host needs the MAC address
of some machine that it has the IP address of. The host broadcasts
an Ethernet frame with an ARP request. The structure of an ARP request
is as follows:
\begin{enumerate}
    \item Hardware type
    \item Protocol type
    \item Hardware size
    \item Protocol size
    \item Opcode
    \item Sender MAC
    \item Target MAC (all 0s)
    \item Target IP
\end{enumerate}

Everyone on a local network gets the
request. If the target IP matches the host
IP, it sends an ARP reply packet.

The structure of an ARP reply is
\begin{enumerate}
    \item Hardware type
    \item Protocol type
    \item Hardware size
    \item Protocol size
    \item Opcode
    \item Sender MAC
    \item Sender IP
    \item Target MAC
    \item Target IP
\end{enumerate}

On getting the ARP reply packet back,
the originating host updates its ARP
table with a new mapping from the target
IP address to target MAC address.

\subsection{Network}

\subsection{Transport}

\subsection{Application}


