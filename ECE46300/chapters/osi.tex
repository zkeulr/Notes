\section{OSI Model}

The Open Systems Interconnection (OSI) model describes
communications from the physical implementation of
transmitting bits across a transmission medium to the
highest-level representation of data of a distributed
application. Each layer has well-defined functions and
semantics and serves a class of functionality to the
layer above it and is served by the layer below it.

\subsection{Physical}

The realm of electrical and computer engineers.
Deals with converting between digital and analog signals or
electrical and optical signals. Beyond the purview of
this course.

\section{Data Link}
The data link layer runs on top of the physical layer.
It transfers data between nodes on a network segment
across the physical layer. Whereas the internet as a whole
runs on a global standard (IP) to allow subnetworks to
communicate, the data link layer allows autonomy
within each local area network (LAN).
Each LAN can run its own network
protocol for communication within LAN,
e.g., Ethernet, Wi-Fi, 5G, CSMA, Sonet, etc.
The data link layers handles addressing,
destination discovery, forwarding, and routing within
a local network. We will study data link layer mechanisms in the
context of the most popular data link layer protocol
called “Ethernet”
Ethernet is an example of a wired data link layer protocol,
i.e., nodes are connected using physical cables
\marginnote{
    Another very popular data link layer protocol is Wi-Fi,
    which is an example of a wireless data link layer protocol.
    Wi-Fi is not covered in this class}

\subsection{Network}
The network layer runs on top of runs on top of the
“best-effort” local area delivery service data link
layer. While the data link layer allows
information to be relayed within a local
network, the network layer connects
different local networks. As with all
layers, we must solve the problems
of addressing, destination discovery,
forwarding, and routing. The problems
are the same, but the scale is different
so the solutions must be different.

\subsection{Transport}
The transport layer provides end-to-end
communication services for applications.
It ensures complete data transfer, error
recovery, and flow control between hosts.
This layer segments and reassembles data
for efficient transmission and provides
reliability with error detection and
correction mechanisms. Common implementations
are \emph{User Datagram Protocol} (UDP) and
\emph{Transmission Control Protocol} (TCP).

\subsection{Application}


