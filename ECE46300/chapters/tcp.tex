\section{Transmission Control Protocol}
Another popular transport layer protocol implementation is
\emph{transmission control protocol}, TCP. TCP is reliable,
bytestream abstracted, and connection-oriented in exactly
the opposite sense of UDP. Reliable means that TCP implements
application multiplexing, ACKs and retransmissions, flow control,
and congestion control. Bytestream is the familiar concept of
communicating using a byte stream abstraction instead of packets.
Connection-oriented means a pairwise sender to receiver connection
is established before sending data and used to maintain connection-specific
state like initial sequence number, window scaling factor, window size, and more.

A TCP header includes the following
fields:
\begin{itemize}
    \item Source port.
    \item Destination port.
    \item Sequence number.
    \item Acknowledgement.
    \item HdrLen, length of the TCP header in number of 4-byte words. Minimum value of 5.
    \item 0.
    \item Flags.
    \item Advertised window.
    \item Checksum. 16 bits for error detection caused by bit corruption.
    \item Urgent pointer.
    \item Options.
\end{itemize}

The checksum for TCP is interested. It's calculated over
a pseudo IP header, which will be explained shortly,
the TCP header, and the payload. The pseudo IP header is
extracted from the IP header and includes the source IP,
the destination IP, a fixed 8 bit string of 0s, the 8-bit
protocol field, and the 16-bit TCP packet length. TCP
connection state is defined using a 5-tuple of source IP,
destination IP, source port, destination port, and protocol,
so the purpose of the pseudo IP header is to ensure the
source and destination IP, plus the protocol fields are not
corrupted.

\subsection{Receive Buffers}
The TCP receiver allocates a receive buffer to hold incoming bytes until the application reads them.
The advertised window field in the TCP header tells the sender how much free space remains in that
buffer and is used for per-connection flow control. TCP uses cumulative ACKs: the acknowledgement
number reports the highest contiguous byte received and delivered to the application. Segments that
arrive out of order can be buffered by the receiver, but they do not advance the cumulative ACK until
the gap is filled; the receiver will commonly send duplicate ACKs (repeating the same acknowledgement
number) to indicate the missing byte range.

On the sender side, transmitted but unacknowledged bytes are kept in a send buffer. Loss detection
uses two mechanisms: duplicate ACKs and timers. Receipt of three duplicate ACKs triggers a fast retransmit
of the missing byte(s) (avoiding the slow timeout path) because repeated ACKs imply a hole in the receiver's
byte stream; waiting for three helps avoid retransmitting for transient reordering. If no ACKs arrive
(for example when a single packet or the final packet is lost) a retransmission timeout (RTO) fires and
the sender retransmits the oldest unacknowledged byte. Most TCP implementations maintain one retransmission
timer per connection, reset it when new data is ACKed, and compute RTO from measured RTTs using conservative
estimates to avoid spurious retransmits. With TCP's cumulative-ACK semantics, retransmission typically follows
a Go-Back-N behavior: the sender restarts transmission from the first unacknowledged byte and proceeds from there.
