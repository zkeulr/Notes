\section{Transmission Control Protocol}
Another popular transport layer protocol implementation is
\emph{transmission control protocol}, TCP. TCP is reliable,
bytestream abstracted, and connection-oriented in exactly
the opposite sense of UDP. Reliable means that TCP implements
application multiplexing, ACKs and retransmissions, flow control,
and congestion control. Bytestream is the familiar concept of
communicating using a byte stream abstraction instead of packets.
Connection-oriented means a pairwise sender to receiver connection
is established before sending data and used to maintain connection-specific
state like initial sequence number, window scaling factor, window size, and more.

A TCP header includes the following
fields:
\begin{itemize}
    \item Source port.
    \item Destination port.
    \item Sequence number.
    \item Acknowledgement.
    \item HdrLen, length of the TCP header in number of 4-byte words. Minimum value of 5.
    \item 0.
    \item Flags.
    \item Advertised window.
    \item Checksum. 16 bits for error detection caused by bit corruption.
    \item Urgent pointer.
    \item Options.
\end{itemize}

The checksum for TCP is interested. It's calculated over
a pseudo IP header, which will be explained shortly,
the TCP header, and the payload. The pseudo IP header is
extracted from the IP header and includes the source IP,
the destination IP, a fixed 8 bit string of 0s, the 8-bit
protocol field, and the 16-bit TCP packet length. TCP
connection state is defined using a 5-tuple of source IP,
destination IP, source port, destination port, and protocol,
so the purpose of the pseudo IP header is to ensure the
source and destination IP, plus the protocol fields are not
corrupted.