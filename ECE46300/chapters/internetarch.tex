\section{Internet Architecture}

We need to solve these problems:
\begin{itemize}
    \item Name and addressing: identifiers for network nodes
    \item Destination discovery: finding the destination address
    \item Forwarding: sending received data to the next hop (neighbor)
    \item Routing: finding the path from source to destination
    \item Reliability: handling failures, packet drops, packet corruption
    \item Application multiplexing: delivering data from multiple host
          applications to the network and vice versa
\end{itemize}

\subsection{Name and Addressing}
Name is the human-readable name for each node (e.g., URL www.google.com).
Address is where node is located (e.g., IP address 172.21.4.110).

\subsection{Destination discovery}
When you go to a web page, you type the name in your browser. You
need to get the address still. The way names are resolved into
addresses is via the Domain Name System (DNS), a set of global
servers that maintain the mappings between host names and addresses.
When you type a URL, the browser contacts a DNS server to get the address.

\subsection{Forwarding}
A router has many ports. Each port in a router acts as both input
and output, i.e., you can both send and receive packets on each port
simultaneously.
When a packet arrives at a router, the router looks at the destination address
in the header. Based on destination address, the router consults the routing
table, determines the right output port, and sends the packet to that port's queue.
For each output port in parallel, when the port is free, the router picks
a packet from the corresponding output queue in some order (e.g. FIFO) and
sends the packet over the output port.

\subsection{Routing}
How do a network of routers collectively find a path between each
source and destination host? The answer is a routing protocol,
a distributed algorithm that runs independently at each router.