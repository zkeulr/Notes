\section{Internet Architecture}

We need to solve these problems:
\begin{itemize}
    \item Name and addressing: identifiers for network nodes
    \item Destination discovery: finding the destination address
    \item Forwarding: sending received data to the next hop (neighbor)
    \item Routing: finding the path from source to destination
    \item Reliability: handling failures, packet drops, packet corruption
    \item Application multiplexing: delivering data from multiple host
          applications to the network and vice versa
\end{itemize}

\subsection{Name and Addressing}
Name is the human-readable name for each node (e.g., URL www.google.com).
Address is where node is located (e.g., IP address 172.21.4.110).

\subsection{Destination discovery}
When you go to a web page, you type the name in your browser. You
need to get the address still. The way names are resolved into
addresses is via the Domain Name System (DNS), a set of global
servers that maintain the mappings between host names and addresses.
When you type a URL, the browser contacts a DNS server to get the address.

\subsection{Forwarding}
A router has many ports. Each port in a router acts as both input
and output, i.e., you can both send and receive packets on each port
simultaneously.
When a packet arrives at a router, the router looks at the destination address
in the header. Based on destination address, the router consults the routing
table, determines the right output port, and sends the packet to that port's queue.
For each output port in parallel, when the port is free, the router picks
a packet from the corresponding output queue in some order (e.g. FIFO) and
sends the packet over the output port.

\subsection{Routing}
How do a network of routers collectively find a path between each
source and destination host? The answer is a routing protocol,
a distributed algorithm that runs independently at each router.

\subsection{Reliability}
The goal of reliability is to ensure every packet sent will
eventually reach the destination uncorrupted. Unfortunately,
routers can drop packets or packets can get corrupted. The idea
behind all reliability protocols is after sending data, await an
ACK from the destination. If ACK was received, everything is good!
If not, resend the data.

\subsection{Application Multiplexing}
Each app that communicates over a network needs some subset of
basic functionalities, like a reliability protocol. We could ask
each app developer to implement the protocol themselves, but this
is burdensome on the developer and prone to errors. Instead, on
modern computers, there is a common OS service that handles
packing data into packets, creating packet headers, and handling ACKs.

Typically, each host has a program running inside the OS called
"host network stack". In addition to the responsibilities above it
also handles getting data from the network and sending it to the right
application.

When an app wants to access the network, it opens a \emph{socket} which
is associated with a \emph{port}. A socket is an OS mechanism that connects
applications to the network stack. A port is a number that specifies a
particular socket. The port number is used by the OS for app multiplexing
to direct incoming packets to the right applications.

\subsection{Modularity}
The internet is inherently hierarchical. There are big networks at high levels,
e.g. AT\&T, which connect smaller networks at a lower hierarchy e.g. Purdue.
We break the task of sending data into five layers, known as the
\emph{Open Systems Interconnect} (OSI) model.

\begin{enumerate}
    \item Physical: bits sent over a physical connection. Involves
          signal processing, analog-to-digital conversions, etc.
    \item Data Link: "best-effort" data delivery within a local network.
          Involves naming, addressing, destination discovery, routing, forwarding
    \item Network: "best-effort" data delivery between networks.
          Involves naming, addressing, destination discovery, routing, forwarding
    \item Transport: reliable end-to-end data delivery.
          Involves reliability and application multiplexing.
    \item Application: network service to users or applications.
          Involves read/write data from applications, encrypt/decrypt, etc.
\end{enumerate}

Each layer only talks with the layers directly above and below.

Where should these subtasks be implemented?