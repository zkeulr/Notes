
\paragraph{MAC Addresses}

All network devices are connected to the network via a
“Network Interface” or “Port”. A network interface can be
“physical” (wired or wireless), such as an actual connection
on a server in some closet, or it can be “virtual”, i.e., a
piece of software emulating a network interface.
Each network interface, physical or virtual, has a Media Access
Control (MAC) address. MAC addresses are 48 bits or 6 bytes long and
typically represented in hexadecimal format, e.g., \texttt{ab:00:05:2c:e4:34}.
MAC address of each interface within a given network must be unique,
but MAC addresses are not necessarily globally unique.

There are three ways to transmit information from a sender to a
recipient:
\begin{itemize}
    \item Unicast: one-to-one transmission
    \item Multicast: many-to-many transmission
    \item Broadcast: one-to-all transmission
\end{itemize}

Naive implementations of
broadcast might have the sender send its packet to every of the
$N-1$ hosts in the network, but a more efficient implementation
is to send the packet to the router and have it send it out
to everyone else.
A special destination MAC address of \texttt{ff:ff:ff:ff:ff:ff} is used
to indicate a broadcast packet.

To see the list of interfaces on your computer, run the following
command in the terminal: \texttt{ifconfig} (mac/Linux) or
\texttt{ipconfig} (Windows).