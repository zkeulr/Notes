\documentclass[nobib]{tufte-handout}

\usepackage{amssymb}
\usepackage{hyperref}
\usepackage{pgfplots}
\usepackage[activate={true,nocompatibility},final,tracking=true,kerning=true,spacing=true,factor=1100,stretch=10,shrink=10]{microtype}
\usepackage{color}
\usepackage{steinmetz}
\usepackage{placeins}
\usepackage{marginfix}
\usepackage{array}
\usepackage{tikz}
\usepackage{amsmath}
\usepackage{amsthm}
\usepackage{booktabs}
\usepackage{listings}
\usepackage[edges]{forest}
\usepackage{caption}
\usepackage[T1]{fontenc}
\usepackage{lmodern}
\usepackage{units}
\usepackage{fancyvrb}
\usepackage{multicol}
\DeclareCaptionFont{white}{\color{white}}
\DeclareCaptionFormat{listing}{\colorbox{gray}{\parbox{\textwidth}{#1#2#3}}}
\captionsetup[lstlisting]{format=listing,labelfont=white,textfont=white}

% Set up the images/graphics package
\usepackage{graphicx}
\setkeys{Gin}{width=\linewidth,totalheight=\textheight,keepaspectratio}
\graphicspath{{.}}

\title{Notes for ECE 46300 - Introduction To Computer Communication Networks }
\author{Zeke Ulrich}
\date{\today} 

\fvset{fontsize=\normalsize}
\usetikzlibrary{shapes}
\usetikzlibrary{positioning}

% For finite state machines 
\usetikzlibrary{automata} % Import library for drawing automata
\usetikzlibrary{positioning} % ...positioning nodes
\usetikzlibrary{arrows} % ...customizing arrows
\tikzset{node distance=2.5cm, % Minimum distance between two nodes. Change if necessary.
    every state/.style={ % Sets the properties for each state
    semithick,
    fill=gray!10},
    initial text={}, % No label on start arrow
    double distance=2pt, % Adjust appearance of accept states
    every edge/.style={ % Sets the properties for each transition
    draw,
    ->,>=stealth', % Makes edges directed with bold arrowheads
    auto,
    semithick}}
\let\epsilon\varepsilon

% These commands are used to pretty-print LaTeX commands
\newcommand{\doccmd}[1]{\texttt{\textbackslash#1}}% command name -- adds backslash automatically
\newcommand{\docopt}[1]{\ensuremath{\langle}\textrm{\textit{#1}}\ensuremath{\rangle}}% optional command argument
\newcommand{\docarg}[1]{\textrm{\textit{#1}}}% (required) command argument
\newenvironment{docspec}{\begin{quote}\noindent}{\end{quote}}% command specification environment
\newcommand{\docenv}[1]{\textsf{#1}}% environment name
\newcommand{\docpkg}[1]{\texttt{#1}}% package name
\newcommand{\doccls}[1]{\texttt{#1}}% document class name
\newcommand{\docclsopt}[1]{\texttt{#1}}% document class option name

% Define a custom command for definitions and biconditional
\newcommand{\defn}[2]{\noindent\textbf{#1}:\ #2}
\let\biconditional\leftrightarrow

\begin{document}

\maketitle

\tableofcontents

\section{Course Description}
An introduction to the design and implementation of computer communication networks. The focus is on the concepts and the fundamental design principles that have contributed to the global Internet success. Topics include: digital transmission and multiplexing, protocols, MAC layer design (Ethernet/802.11), LAN interconnects and switching, congestion/flow/error control, routing, addressing, performance evaluation, internetworking (Internet) including TCP/IP, HTTP, DNS etc. This course will include one or more programming projects.
\pagebreak

\section{Introduction}
Let's examine some basic C++ programs.
\begin{lstlisting}[language=C++,caption=Hello World]
    #include <iostream>

    int main() {
        std::cout << "Hello World!";
        return 0;
    }
\end{lstlisting}

\begin{lstlisting}[language=C++,caption=User Input]
    #include <iostream>

    int main() {
        double n;
        int i;

        std::cout << "Enter float: ";
        std::cin >> n;

        std::cout << "Enter integer: ";
        std::cin >> i;

        return 0;
    }
\end{lstlisting}

The \texttt{stds} you're seeing all over the place refer not
to a frat party but the standard namespace. It holds
useful objects like standard in ("\texttt{cin}") and standard
out ("\texttt{cout}").

In C, we use \texttt{malloc} and \texttt{free} to allocate memory. In C++,
the equivalent operations are \texttt{new} and \texttt{delete}.

\begin{lstlisting}[language=C++,caption=Free and Delete]
    void CorrectUsage(){
        int *ptr = new int[3];
        int *ptr1 = new int;
        ptr[0] = 1;
        ptr[1] = 2;
        ptr[2] = 3;
        *ptr1 = 5;
        delete ptr1;
        delete [] ptr;        
\end{lstlisting}
The reason for this is using \texttt{new} and \texttt{free} calls
an object's constructor and destructors, which are defined to properly
delete the object. Technically, \texttt{malloc} and \texttt{free} are
both present in C++, but they won't trigger the constructors and
destructors and should be avoided. The choice to leave these functions in
was made to improve compatibility with C, which is a theme the reader
may notice in C++'s many possible ways to do the same thing.

C++ filenames are terminated with a \texttt{.cpp} or \texttt{.cc}
extension, like \texttt{example.cpp} or \texttt{example.cc}


\end{document}