\documentclass[nobib]{tufte-handout}

%\\geometry{showframe}% for debugging purposes -- displays the margins

\newcommand{\bra}[1]{\left(#1\right)}
\usepackage{clrscode3e}
\usepackage{hyperref}
\usepackage[activate={true,nocompatibility},final,tracking=true,kerning=true,spacing=true,factor=1100,stretch=10,shrink=10]{microtype}
\usepackage{color}

\usepackage{tikz}
\usepackage{amsmath,amsthm}
\usetikzlibrary{shapes}
\usetikzlibrary{positioning}

% Set up the images/graphics package
\usepackage{graphicx}
\setkeys{Gin}{width=\linewidth,totalheight=\textheight,keepaspectratio}
\graphicspath{{.}}

\title{Notes for PHYS 27200 - Electric And Magnetic Interactions}
\author[Ezekiel Ulrich]{Ezekiel Ulrich}
\date{\today}  % if the \date{} command is left out, the current date will be used

% The following package makes prettier tables.  We're all about the bling!
\usepackage{booktabs}

% The units package provides nice, non-stacked fractions and better spacing
% for units.
\usepackage{units}

% The fancyvrb package lets us customize the formatting of verbatim
% environments.  We use a slightly smaller font.
\usepackage{fancyvrb}
\fvset{fontsize=\normalsize}

% Small sections of multiple columns
\usepackage{multicol}

% These commands are used to pretty-print LaTeX commands
\newcommand{\doccmd}[1]{\texttt{\textbackslash#1}}% command name -- adds backslash automatically
\newcommand{\docopt}[1]{\ensuremath{\langle}\textrm{\textit{#1}}\ensuremath{\rangle}}% optional command argument
\newcommand{\docarg}[1]{\textrm{\textit{#1}}}% (required) command argument
\newenvironment{docspec}{\begin{quote}\noindent}{\end{quote}}% command specification environment
\newcommand{\docenv}[1]{\textsf{#1}}% environment name
\newcommand{\docpkg}[1]{\texttt{#1}}% package name
\newcommand{\doccls}[1]{\texttt{#1}}% document class name
\newcommand{\docclsopt}[1]{\texttt{#1}}% document class option name

% Define a custom command for definitions
\newcommand{\defn}[2]{\textbf{#1}:\ #2}

\begin{document}

\maketitle

\begin{abstract}
These are lecture notes for fall 2023 PHYS 27200 at Purdue. Modify, use, and distribute as you please.
\end{abstract}

\tableofcontents

\section{Course Introduction}

This is a calculus-based physics course using concepts of electric and magnetic fields and an atomic description of
matter to describe polarization, fields produced by charge distributions, potential, electrical circuits,
magnetic forces, induction, and related topics, leading to Maxwell's equations and electromagnetic
radiation and an introduction to waves and interference. 3-D graphical simulations and numerical problem
solving by computer are employed throughout. For more information, consult the syllabus.

\section{Equations}

\begin{enumerate}
    \item Coulomb's Law: $\vec{F} = \frac{1}{4\pi \epsilon_0}\frac{q_1 q_2}{r^2}\hat{r}$
    \item $\vec{E_1} = \frac{1}{4\pi \epsilon_0}\frac{q_1}{r^2}\hat{r}$
    \item $\vec{F_2} = E_1 q_2$
    \item Dipole moment between charges $-q$ and $q$ separated by $\vec{s}$: $\vec{p} = q\vec{s}$

\end{enumerate}

\section{Electric charge}

\defn{Electric Charge}{Electric charge is an intrinsic characteristic of the
fundamental particles that make up objects.}

Object can hae negative, zero, or postive charge. 

\section{Electric field}

\begin{marginfigure}
    \includegraphics{electricfieldlines.jpg}
    \caption{An eletric field coming from point charges. 
    Notice how the densities of the lines vary with distance from the source.}
    \label{fig:electric-field-lines-point-charge}
\end{marginfigure}

We can represent eletric fields as lines emenating from a point charge. The greater the density of the lines, the greater 
the strength of the electric field. Note that at the origin, the force is
undefined (infinite), since $|r| = 0$. 

Consider the relative strengths of the electric and 
gravitational fields. The gravitational force is given by
$F_g = G\frac{m_1m_2}{r^2}\hat{r}$, with $m_{electron} = 9 \times 10^{-31} kg$
and $m_{proton} = 1.7 \times 10^{-27} kg$. If we consider a hydrogen atom,  
then $r = 5.3 \times 10^{-11} m$. With $G = 6.7 \times 10^{-11}$, we have 
\[F_g = \frac{(1.7 \times 10^{-27})(9 \times 10^{-31})(6.7 \times 10^{-11})}{(5.3 \times 10^{-11})^2} \approx O(10^{-46})N\]
Now, the eletric force is given by $\vec{F} = \frac{1}{4\pi \epsilon_0}\frac{q_1 q_2}{r^2}\hat{r}$. 
The charge of a proton and eletric are $q_1 \approx q_2 \approx 1.6 \times 10^{-19} C$.
Ergo, since $\frac{1}{4\pi \epsilon_0} \approx 8.99 \times 10^9 \frac{Nm^2}{C^2}$, 
\[F_e = \frac{(8.99 \times 10^9 \frac{Nm^2}{C^2})(1.60\times10^{-19}C)^2}{(5.3\times10^{-11}m)^2} \approx O(10^-17)N\]
This means that $\frac{F_e}{F_g}\approx 2.27 \times 10^{39}$, meaning the eletric force is
much stronger for these masses and charges than gravity. On scales as large as humans and planets,
gravity is the dominant force because gravity is strictly additive.

For sufficient distances, the eletric field
of a uniformly charged spherical shell resembles
the eletric field of a point charge. 

\begin{marginfigure}
    \begin{center}
    \begin{tikzpicture}
        % Dot
        \fill (0,0) circle (1pt);
        % Arrow
        \draw[->] (0,0) -- (0,3);
    \end{tikzpicture}
    \end{center}

    \caption{Notice how a circle resembles a point from a great distance.}
    \label{fig:long-distance-point-sphere}
\end{marginfigure}

That means for $r>>R$, $\vec{E_{sphere}} = \frac{1}{4\pi \epsilon_0}\frac{q_1}{r^2}\hat{r}$
This holds only for outside the sphere. Inside, it can be shown that the electric field is zero. 

\defn{Superposition Principle}{The net electric field at a location in space is a
vector sum of the individual electric fields contributed by all
charged particles located elsewhere.}

To introduce systems with multiple sources of electric field lines, consider
the particle pair known as a dipole. Dipoles consist of one negatively charged
and one positively charged particle, like so:
\begin{marginfigure}
    \centering
    \includegraphics[width=\textwidth / 2]{VFPt_dipole_electric.svg.png}
    \caption{Two oppositely charged particles distanced from one another}
    \label{fig:dipole}
\end{marginfigure}

\defn{Dipole Moment}{The dipole moment is a way of expressing asymmetrical charge distribution.
It is a vector quantity, i.e. it has magnitude as well as definite directions.}

On the axis of the dipole (i.e. the lines
formed by the two particles), the electric field is 
\[\vec{E} = \frac{1}{4\pi \epsilon_0}\frac{2sq}{r^3}\hat{p}\]
On the bisecting plane (i.e., the plane exactly halfway from each 
point) the field is given by
\[\vec{E} = \frac{-1}{4 \pi \epsilon_0}\frac{sq}{r^3}\hat{p}\]

\marginnote{}


\end{document}
