\documentclass[nobib]{tufte-handout}

\newcommand{\bra}[1]{\left(#1\right)}
\usepackage{hyperref}
\usepackage[activate={true,nocompatibility},final,tracking=true,kerning=true,spacing=true,factor=1100,stretch=10,shrink=10]{microtype}
\usepackage{color}

% Set up the images/graphics package
\usepackage{graphicx}
\setkeys{Gin}{width=\linewidth,totalheight=\textheight,keepaspectratio}

\title{Notes for POL 23700 - Modern Weapons And International Relations}
\author[Zeke Ulrich]{Zeke Ulrich}
\date{\today}  % if the \date{} command is left out, the current date will be used

% The following package makes prettier tables.  We're all about the bling!
\usepackage{booktabs}

% The units package provides nice, non-stacked fractions and better spacing
% for units.
\usepackage{units}

% The fancyvrb package lets us customize the formatting of verbatim
% environments.  We use a slightly smaller font.
\usepackage{fancyvrb}
\fvset{fontsize=\normalsize}

% Small sections of multiple columns
\usepackage{multicol}

% These commands are used to pretty-print LaTeX commands
\newcommand{\doccmd}[1]{\texttt{\textbackslash#1}}% command name -- adds backslash automatically
\newcommand{\docopt}[1]{\ensuremath{\langle}\textrm{\textit{#1}}\ensuremath{\rangle}}% optional command argument
\newcommand{\docarg}[1]{\textrm{\textit{#1}}}% (required) command argument
\newenvironment{docspec}{\begin{quote}\noindent}{\end{quote}}% command specification environment
\newcommand{\docenv}[1]{\textsf{#1}}% environment name
\newcommand{\docpkg}[1]{\texttt{#1}}% package name
\newcommand{\doccls}[1]{\texttt{#1}}% document class name
\newcommand{\docclsopt}[1]{\texttt{#1}}% document class option name

% Define a custom command for definitions
\newcommand{\defn}[2]{\noindent\textbf{#1}:\ #2}

% Define graphics path
\graphicspath{ {./images/} }

\begin{document}

\maketitle

\begin{abstract}
These are lecture notes for fall 2023 POL 23700 at Purdue.
\end{abstract}

\tableofcontents

\section{Course Introduction}
This course introduces the student to the roles that modern weapons systems 
play in contemporary international relations.
\\~\\
Learning objectives: 
\begin{enumerate}
    \item Identify and explain the elements and requirements of nuclear deterrence.
    \item Discuss the role of technology in the emergence of modern total warfare.
    \item Analyze the impact of contemporary information technologies on the conduct of warfare.
\end{enumerate}

\pagebreak 

\section{Military revolutions}
\defn{RMA}{Revolution in military affairs. A major change in warfare brought about by a new application of technology.}
\defn{Total war}{The full mobilization of a country for the war effort, where 
the entire economy, social organization, and civil population become dedicated 
to the war effort.}

Technology is the great equalizer. Whereas historically
power was held by trained warriors and those who 
commanded them, the democratization of force through 
modern war machines enables a nineteen-year-old boot camp 
graduate to have the same effect on the battlefield
as a soldier with decades of experience. 

The five most important RMAs are, in chronological order:
\begin{enumerate}
    \item The gunpowder revolution
    \item The Napoleonic revolution
    \item The industrial revolution
    \item The airpower revolution
    \item The nuclear revolution
\end{enumerate}
In general, these things are true of RMAs:
\begin{itemize}
    \item They involve new technologies
    \item Technology is not limited to Weapons
    \item Strategic competition encourages military innovation
    \item Innovation in warfare is driven by the basic struggle of defense vs offense
    \item RMAs are driver by technology, which is self-accelerating. Thus each RMA occurs faster than the previous
\end{itemize}

\begin{marginfigure}
    \includegraphics{Coevorden.jpg}
    \caption{Bastion fort}
    \label{bastion}
\end{marginfigure}
\marginpar{
    The bastion fort was a very flat structure composed of 
    many triangular bastions, specifically designed to cover each 
    other. To counteract cannonballs, defensive 
    walls were made stouter.
}

\subsection{The gunpowder revolution}
The gunpowder revolution lasted from the 1400s to the 1700s.
Prior political power was decentralized amongst 
smaller localities. In Europe, these smaller localities were hundreds of lords 
guided by the overarching influence of the Catholic church. Defense had the advantage. 
Sieges could last months or years, allowing the defenders an ever-present option to retreat. 
Knights were the dominant power. The more numerous footmen were untrained peasants 
pressed into service by nobles. Between the 1400s and 1850 consolidated countries emerged,
largely thanks to newly invented cannons capable of destroying castle walls. 
Defensive attempts to mitigate the destructive power of cannons, such as bastion forts, were expensive
and rare. Now that cannons were able to easily destroy castles, 
royalty needed a strong, constant military force to protect themselves. 
These trained armies were able to combat the poorly organized feudal knights 
and led to the solidification of nation states. Feudal states, independent cities, 
and religious enclaves had no ability to forward standing armies and were conquered 
and assimilated. The battlefield advantage shifted from skilled knights to
masses of peasants taught to hold a gun straight and fire on command. 
Discipline became more important than skill. Lines of riflemen 
faced off, firing and firing again 
until enough musket balls had found their mark that the opposing line 
fell. 

As states with these larger armies assimilated their 
neighbors, Europe as we know it today began 
to emerge. Power solidified within families 
and individuals, and the medieval era of loose organization 
was supplanted by one of tighter regulation and control. 

\subsection{The Napoleonic revolution}

When Napoleon emerged as a great leader, the world 
experienced true nationalism for the first time. No 
longer did the poor need to be pressed into service, 
but now could be coerced in service of a greater cause. 
These patriotic soldiers were more motivated and 
their compatriots at home more willing to support 
the war. Citizens of French towns began thinking of 
themselves as French citizens, and national pride 
began to spread outwards from France. Napoleon structured his 
army into self-sufficient and purpose-built corps who supported 
one another. He promoted based on skill, improved logistics, 
and conquered large swathes of Europe before his enemies learned 
to adopt or counter his tactics. Angered by French conquest, 
nationalism flared in Prussia, Spain, and Britain and 
the disjoint European nations banded together to oppose the 
superiority of Napoleonic France. We thus learn an important lesson:
good technology is adopted by everyone, eventually reducing the 
exclusivity its inventors at first enjoy. This lesson repeats itself 
in every RMA we will study. 

\subsection{The industrial revolution}

The industrial revolution allowed for the first true instance of total war, 
as states centralized after the Napoleonic wars became able to direct 
output and civilian population to drive large-scale wars. Three important 
technologies emerged under the industrial revolution:
\begin{itemize}
    \item The railroad
    \item The telegraph
    \item The rifle
\end{itemize}
These innovations will be crucial going forward. Railroads allowed 
the mass transit of supplies and troops across land, allowing massive 
and rapid mobilization and effectively bringing countries closer together. 
Speed and response time became much more important once an enemy army could travel across 
several countries in the span of hours. Efficient and timely execution of 
orders necessitated efficient and timely communication of orders, 
which came with the invention of the telegraph. The telegraph sped 
up war even farther and shifted the advantage to the first mover in a war,
such as the Prussians in the wars of German reunification. 

Now, let's look at the final crucial invention: the rifle. 
Old muskets took half a minute to load under the best conditions, 
malfunctioned twenty-five percent of the time, and had 
an accurate range of around one hundred meters. Newly 
invented percussion rifles could kill at a mile, was much 
more reliable, and once repeating rifles were invented, 
soldiers could fire dozens of rounds per minute. The logical 
conclusion of this accelerating firing speed was the machine gun, 
capable of firing hundreds of rounds per minute. 
While the gunpowder revolution made firearms possible, 
each one had to be hand made and assembled. The processes 
of the industrial revolution made guns cheaper and faster, 
making guns accessible, interchangeable, and more common. 

The tinderbox of heavily armed, nationalist European states 
was quick to ignite when the Serbian assassination of Austrian archduke 
Franz Ferdinand sparked WWI. Military thinkers of the time 
saw the success of the wars of German reunification 
and believed that victory would favor those who struck first.
The Germans planned, in the event of war, to strike France 
by bypassing its defense through Belgium, then shifting 
attention to the lethargic beast of Russia, before fending 
off any attack from Great Britain, who would be threatened 
by the proximity of Belgian ports to the English coastline. 
After Franz Ferdinand's assassination, Germany struck first
and launched an invasion of Belgium. Thanks to the railroad 
and telegram, France was able to quickly mobilize inwards and 
repel the German invasion miles outside of Paris. While fleeing, 
Germans turned around, dug trenches, set up machine guns, 
and fired backwards on the advancing French forces. Unable 
to advance, French and German forces tried to outflank 
one another, lengthening the trenches until they spanned the entire country. 
Nationalism drove thousands of young enlists to the trenches, 
where they shot one another with mass-produced rifles. 
Movement at the trenches stalled, causalities mounted, 
and poison gas floated across no-man's land to dissolve the 
lungs of young recruits. Primitive airplanes were unable to 
advance the front, and progress seemed impossible until 
the invention of modern warfare's most iconic children: 
the tank. 

The tank, along with the entry of the United States late 
in the war, broke through the German lines. Germany was routed 
as the tank could survive the machine gun and small-arms 
fire in no man's land, travel over difficult terrain, 
crush barbed wire, and cross trenches to assault 
fortified enemy positions with powerful armament. However, 
technology again diffused, and by WWII every major country had 
adopted this powerful war machine, again equalizing the 
battlefield. Germany's WWII Blitzkrieg doctrine combined intensive training,
wireless radio communication, and novel tactics, allowing it 
to steamroll Poland and France. The Nazis used radio communication to 
exploit weaknesses in enemy lines, create holes, push through, 
and respond rapidly. 

\subsection{The airpower revolution}
\subsection{The nuclear revolution}

\end{document}