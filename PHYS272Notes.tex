\documentclass[nobib]{tufte-handout}

%\\geometry{showframe}% for debugging purposes -- displays the margins

\newcommand{\bra}[1]{\left(#1\right)}
\usepackage{clrscode3e}
\usepackage{hyperref}
\usepackage[activate={true,nocompatibility},final,tracking=true,kerning=true,spacing=true,factor=1100,stretch=10,shrink=10]{microtype}
\usepackage{color}

\usepackage{tikz}
\usepackage{amsmath,amsthm}
\usetikzlibrary{shapes}
\usetikzlibrary{positioning}
% Set up the images/graphics package
\usepackage{graphicx}
\setkeys{Gin}{width=\linewidth,totalheight=\textheight,keepaspectratio}
\graphicspath{{.}}

\title{Notes for ECE 26400 - Advanced C Programming}
\author[Ezekiel Ulrich]{Ezekiel Ulrich}
\date{\today}  % if the \date{} command is left out, the current date will be used

% The following package makes prettier tables.  We're all about the bling!
\usepackage{booktabs}

% The units package provides nice, non-stacked fractions and better spacing
% for units.
\usepackage{units}

% The fancyvrb package lets us customize the formatting of verbatim
% environments.  We use a slightly smaller font.
\usepackage{fancyvrb}
\fvset{fontsize=\normalsize}

% Small sections of multiple columns
\usepackage{multicol}

% These commands are used to pretty-print LaTeX commands
\newcommand{\doccmd}[1]{\texttt{\textbackslash#1}}% command name -- adds backslash automatically
\newcommand{\docopt}[1]{\ensuremath{\langle}\textrm{\textit{#1}}\ensuremath{\rangle}}% optional command argument
\newcommand{\docarg}[1]{\textrm{\textit{#1}}}% (required) command argument
\newenvironment{docspec}{\begin{quote}\noindent}{\end{quote}}% command specification environment
\newcommand{\docenv}[1]{\textsf{#1}}% environment name
\newcommand{\docpkg}[1]{\texttt{#1}}% package name
\newcommand{\doccls}[1]{\texttt{#1}}% document class name
\newcommand{\docclsopt}[1]{\texttt{#1}}% document class option name

\begin{document}

\maketitle

\begin{abstract}
These are lecture notes for fall 2023 ECE 26400 at Purdue. Modify, use, and distribute as you please.
\end{abstract}

\tableofcontents

\section{Course Introduction}

\section{Equations}

\begin{enumerate}
    \item $\vec{F} = \frac{1}{4\pi \epsilon_0}\frac{q_1 q_2}{r^2}\hat{r}$
    \item $\vec{E_1} = \frac{1}{4\pi \epsilon_0}\frac{q_1}{r^2}\hat{r}$
    \item $\vec{F_2} = E_1 q_2$

\end{enumerate}

\section{Eletric field}

We can represent eletric fields as lines emenating from a point charge. The greater the density of the lines, the greater 
the strength of the electric field. Note that at the origin, the force is
undefined (infinite), since $|r| = 0$. 

Consider the relative strengths of the electric and 
gravitational fields. The gravitational force is given by
$F_g = G\frac{}{}$

\begin{marginfigure}
    \includegraphics{eletricfieldlines.png}
    \caption{An eletric field coming from point charges. 
    Notice how the densities of the lines vary with distance from the source.}
    \label{fig:electric-field-lines-point-charge}
\end{marginfigure}

\marginnote{}

\end{document}
