\section{Introduction}
An algorithm is simply a method
to solve a class of problems.
There are algorithms to make
toast, to solve a Rubik's cube,
and to plot the optimal path
for a mailmain.
The two most important qualities
of any algorithm are effectiveness
and secondly efficiency. Our solution
must work, preferably in as little
time as possible. A working algorithm
is no good if the heat death of the
universe occurs before it finds a
solution.
We describe the cost of an algorithm
using Big O notation. As an example,
consider listing \ref{lst:nsum}
\begin{figure}
    \begin{lstlisting}
        int total = 0; // C_1
        for (int i = 0; i++; i <= n){ // C_2
            total = total + i; // C_3
        }
        return total; // C_4
    \end{lstlisting}
    \caption{Sum of first $n$ numbers}
    \label{lst:nsum}
\end{figure}
If each operation has cost $C_i$,
then the total cost of the program is
\begin{equation}
    C_1 + C_2 (n + 1) + C_3 n + C_4 = n(C_2 + C_3) + (C_1 + C_4).
\end{equation}
Big O notation considers only
the largest power of $n$, so the Big O complexity
for this algorithm would be $O(n)$.

A data structure is an
organization of information for
ease of manipulation. For
example, a dictionary,
a checkout line, and an
org chart.
