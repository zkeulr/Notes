\section{Sorting}
\subsection{Bubble Sort}
Bubble Sort is a simple sorting algorithm that repeatedly steps 
through the list, compares adjacent elements, and swaps them if 
they are in the wrong order. The pass through the list is repeated 
until the list is sorted.

\textbf{Time Complexity:}
\begin{itemize}
    \item Best Case: $O(n)$ (when the array is already sorted)
    \item Average Case: $O(n^2)$
    \item Worst Case: $O(n^2)$
\end{itemize}

\begin{lstlisting}[caption=Bubble Sort]
void bubble_sort(int arr[], int n) {
    for (int i = 0; i < n-1; i++) {
        int swapped = 0;
        for (int j = 0; j < n-i-1; j++) {
            if (arr[j] > arr[j+1]) {
                int temp = arr[j];
                arr[j] = arr[j+1];
                arr[j+1] = temp;
                swapped = 1;
            }
        }
        if (swapped == 0)
            break;
    }
}
\end{lstlisting}
\subsection{Insertion Sort}
Insertion Sort builds the sorted array one item at a time. 
It picks an element and places it into its correct position in 
the sorted part of the array.

\textbf{Time Complexity:}
\begin{itemize}
    \item Best Case: $O(n)$ (when the array is already sorted)
    \item Average Case: $O(n^2)$
    \item Worst Case: $O(n^2)$
\end{itemize}

\begin{lstlisting}[caption=Insertion Sort]
void insertion_sort(int arr[], int n) {
    for (int i = 1; i < n; i++) {
        int key = arr[i];
        int j = i - 1;
        while (j >= 0 && arr[j] > key) {
            arr[j + 1] = arr[j];
            j = j - 1;
        }
        arr[j + 1] = key;
    }
}
\end{lstlisting}
\subsection{Selection Sort}
Selection Sort divides the array into a sorted and an unsorted 
region. It repeatedly selects the smallest (or largest) element 
from the unsorted region and moves it to the sorted region.

\textbf{Time Complexity:}
\begin{itemize}
    \item Best Case: $O(n^2)$
    \item Average Case: $O(n^2)$
    \item Worst Case: $O(n^2)$
\end{itemize}

\begin{lstlisting}[caption=Selection Sort]
void selection_sort(int arr[], int n) {
    for (int i = 0; i < n-1; i++) {
        int min_idx = i;
        for (int j = i+1; j < n; j++) {
            if (arr[j] < arr[min_idx])
                min_idx = j;
        }
        int temp = arr[min_idx];
        arr[min_idx] = arr[i];
        arr[i] = temp;
    }
}
\end{lstlisting}
\subsection{Shell Sort}
Shell Sort is an optimization over Insertion Sort. 
It first sorts elements that are far apart and then progressively 
reduces the gap between elements to be sorted.

\textbf{Time Complexity:}
\begin{itemize}
    \item Best Case: $O(n \log n)$
    \item Average Case: depends on the gap sequence, commonly $O(n^{3/2})$ or $O(n \log^2 n)$
    \item Worst Case: $O(n^2)$
\end{itemize}

\begin{lstlisting}[caption=Shell Sort]
void shell_sort(int arr[], int n) {
    for (int gap = n/2; gap > 0; gap /= 2) {
        for (int i = gap; i < n; i++) {
            int temp = arr[i];
            int j;
            for (j = i; j >= gap && arr[j - gap] > temp; j -= gap) {
                arr[j] = arr[j - gap];
            }
            arr[j] = temp;
        }
    }
}
\end{lstlisting}

A sorting algorithm is stable if two objects with equal keys 
appear in the same order in sorted output as they appear in 
the input array to be sorted. Insertion sort, merge sort, and 
bubble sort are all stable sorting algorithms. 