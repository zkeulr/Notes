\section{Asymptotic Notation}

\subsection{Big O}
Big $O$ notation describes the upper bound 
of an algorithm's growth rate. It provides an 
asymptotic measure of the time complexity or 
space complexity of an algorithm as a function of 
the input size. Informally, big $O$ notation
just picks out the highest 
order term from your runtime expression. 
Formally, Big $O$ is defined as follows:
\[
f(n) \in O(g(n)) \iff \exists \, c > 0 \text{ and } n_0 \geq 0 \text{ such that } \forall n \geq n_0, \, 0 \leq f(n) \leq c \cdot g(n).
\]
\begin{itemize}
    \item $f(n)$: The function representing the actual growth rate of the algorithm.
    \item $g(n)$: The comparison function, often representing the dominant term of $f(n)$.
    \item $c > 0$: A constant that scales $g(n)$ to bound $f(n)$.
    \item $n_0$: A constant representing the threshold beyond which the inequality holds.
\end{itemize}

Big $O$ notation focuses on the dominant term of $g(n)$, 
ignoring lower-order terms and constant factors, to provide a 
simplified representation of the algorithm's efficiency.

\subsection{Big $\Omega$}
Big $\Omega$ notation describes the lower bound 
of an algorithm's growth rate. It provides an 
asymptotic measure of the minimum time complexity or 
space complexity of an algorithm as a function of 
the input size. While Big $O$ focuses on the upper bound, 
Big $\Omega$ focuses on the lower bound, ensuring that the 
algorithm's growth rate is at least as fast as a specified 
function, up to constant factors. Formally, Big $\Omega$ is defined as follows:
\[
f(n) \in \Omega(g(n)) \iff \exists \, c > 0 \text{ and } n_0 \geq 0 \text{ such that } \forall n \geq n_0, \, 0 \leq c \cdot g(n) \leq f(n).
\]
\begin{itemize}
    \item $f(n)$: The function representing the actual growth rate of the algorithm.
    \item $g(n)$: The comparison function, representing the lower bound on $f(n)$.
    \item $c > 0$: A constant that scales $g(n)$ to bound $f(n)$ from below.
    \item $n_0$: A constant representing the threshold beyond which the inequality holds.
\end{itemize}

Big $\Omega$ notation is used to describe the best-case or 
minimum growth rate of an algorithm. For example, if an 
algorithm's runtime is $\Omega(n \log n)$, it means that the 
algorithm cannot perform better than $n \log n$ in the best case, 
up to constant factors. This notation is particularly useful 
when proving lower bounds on the complexity of problems or algorithms.

\subsection{Big $\Theta$}
Big $\Theta$ notation describes both the upper and lower bounds 
of an algorithm's growth rate. It provides a tight asymptotic 
bound on the time complexity or space complexity of an algorithm 
as a function of the input size. Unlike Big $O$, which only provides 
an upper bound, Big $\Theta$ ensures that the growth rate of the 
algorithm is bounded both above and below by the same function, 
up to constant factors. Formally, Big $\Theta$ is defined as follows:
\[
f(n) \in \Theta(g(n)) \iff \exists \, c_1, c_2 > 0 \text{ and } n_0 \geq 0 \text{ such that } \forall n \geq n_0, \, c_1 \cdot g(n) \leq f(n) \leq c_2 \cdot g(n).
\]
\begin{itemize}
    \item $f(n)$: The function representing the actual growth rate of the algorithm.
    \item $g(n)$: The comparison function, representing the tight bound on $f(n)$.
    \item $c_1, c_2 > 0$: Constants that scale $g(n)$ to bound $f(n)$ from below and above.
    \item $n_0$: A constant representing the threshold beyond which the inequality holds.
\end{itemize}

Big $\Theta$ notation is used when the growth rate of an algorithm 
is known to be tightly constrained by a specific function, providing 
a precise characterization of its efficiency. For example, if an 
algorithm's runtime is both $O(n^2)$ and $\Omega(n^2)$, then its 
runtime is $\Theta(n^2)$.

\subsection{Little $o$}
Little $o$ notation describes an upper bound that is not tight. 
It provides a stricter asymptotic measure of the growth rate 
of an algorithm compared to Big $O$ notation. While Big $O$ 
allows for the possibility that the growth rates are the same 
up to constant factors, Little $o$ ensures that the growth rate 
of the function is strictly less than the comparison function. 
Formally, Little $o$ is defined as follows:
\[
f(n) \in o(g(n)) \iff \forall \, c > 0, \, \exists \, n_0 \geq 0 \text{ such that } \forall n \geq n_0, \, 0 \leq f(n) < c \cdot g(n).
\]
\begin{itemize}
    \item $f(n)$: The function representing the actual growth rate of the algorithm.
    \item $g(n)$: The comparison function, representing the upper bound on $f(n)$.
    \item $c > 0$: A constant that scales $g(n)$ to bound $f(n)$ strictly from above.
    \item $n_0$: A constant representing the threshold beyond which the inequality holds.
\end{itemize}

Little $o$ notation is used to describe situations where the growth rate 
of an algorithm is strictly smaller than the comparison function. For example, 
if $f(n) = 2n + 3$, then $f(n) \in o(n^2)$, because $2n + 3$ grows strictly 
slower than $n^2$ as $n$ becomes large. This notation is particularly useful 
when emphasizing that an algorithm's growth rate is asymptotically insignificant 
compared to another function.

\subsection{Little $\omega$}
Little $\omega$ notation describes a lower bound that is not tight. 
It provides a stricter asymptotic measure of the growth rate 
of an algorithm compared to Big $\Omega$ notation. While Big $\Omega$ 
allows for the possibility that the growth rates are the same 
up to constant factors, Little $\omega$ ensures that the growth rate 
of the function is strictly greater than the comparison function. 
Formally, Little $\omega$ is defined as follows:
\[
f(n) \in \omega(g(n)) \iff \forall \, c > 0, \, \exists \, n_0 \geq 0 \text{ such that } \forall n \geq n_0, \, 0 \leq c \cdot g(n) < f(n).
\]
\begin{itemize}
    \item $f(n)$: The function representing the actual growth rate of the algorithm.
    \item $g(n)$: The comparison function, representing the lower bound on $f(n)$.
    \item $c > 0$: A constant that scales $g(n)$ to bound $f(n)$ strictly from below.
    \item $n_0$: A constant representing the threshold beyond which the inequality holds.
\end{itemize}

Little $\omega$ notation is used to describe situations where the growth rate 
of an algorithm is strictly larger than the comparison function. For example, 
if $f(n) = n^2$, then $f(n) \in \omega(n)$, because $n^2$ grows strictly 
faster than $n$ as $n$ becomes large. This notation is particularly useful 
when emphasizing that an algorithm's growth rate is asymptotically dominant 
compared to another function.