\documentclass[nobib]{tufte-handout}

%\\geometry{showframe}% for debugging purposes -- displays the margins

\newcommand{\bra}[1]{\left(#1\right)}
\usepackage{amssymb}
\usepackage{hyperref}
\usepackage{pgfplots}
\usepackage[activate={true,nocompatibility},final,tracking=true,kerning=true,spacing=true,factor=1100,stretch=10,shrink=10]{microtype}
\usepackage{color}
\usepackage{steinmetz}
\usepackage{placeins}
% Fixes captions and images being cut off
\usepackage{marginfix}
\usepackage{array}
\usepackage{tikz}
\usepackage{amsmath,amsthm}
\usetikzlibrary{shapes}
\usetikzlibrary{positioning}
\usepackage{listings}
\usepackage{forest}
\usepackage{caption}
\DeclareCaptionFont{white}{\color{white}}
\DeclareCaptionFormat{listing}{\colorbox{gray}{\parbox{\textwidth}{#1#2#3}}}
\captionsetup[lstlisting]{format=listing,labelfont=white,textfont=white}

% Set up the images/graphics package
\usepackage{graphicx}
\setkeys{Gin}{width=\linewidth,totalheight=\textheight,keepaspectratio}
\graphicspath{{.}}

\title{Notes for ECE 27000 - Introduction to Digital System Design}
\author{Zeke Ulrich}
\date{\today}  % if the \date{} command is left out, the current date will be used

% The following package makes prettier tables.  We're all about the bling!
\usepackage{booktabs}

% The units package provides nice, non-stacked fractions and better spacing
% for units.
\usepackage{units}

% The fancyvrb package lets us customize the formatting of verbatim
% environments.  We use a slightly smaller font.
\usepackage{fancyvrb}
\fvset{fontsize=\normalsize}

% Small sections of multiple columns
\usepackage{multicol}

% For finite state machines 
\usetikzlibrary{automata} % Import library for drawing automata
\usetikzlibrary{positioning} % ...positioning nodes
\usetikzlibrary{arrows} % ...customizing arrows
\tikzset{node distance=2.5cm, % Minimum distance between two nodes. Change if necessary.
    every state/.style={ % Sets the properties for each state
    semithick,
    fill=gray!10},
    initial text={}, % No label on start arrow
    double distance=2pt, % Adjust appearance of accept states
    every edge/.style={ % Sets the properties for each transition
    draw,
    ->,>=stealth', % Makes edges directed with bold arrowheads
    auto,
    semithick}}
\let\epsilon\varepsilon

% These commands are used to pretty-print LaTeX commands
\newcommand{\doccmd}[1]{\texttt{\textbackslash#1}}% command name -- adds backslash automatically
\newcommand{\docopt}[1]{\ensuremath{\langle}\textrm{\textit{#1}}\ensuremath{\rangle}}% optional command argument
\newcommand{\docarg}[1]{\textrm{\textit{#1}}}% (required) command argument
\newenvironment{docspec}{\begin{quote}\noindent}{\end{quote}}% command specification environment
\newcommand{\docenv}[1]{\textsf{#1}}% environment name
\newcommand{\docpkg}[1]{\texttt{#1}}% package name
\newcommand{\doccls}[1]{\texttt{#1}}% document class name
\newcommand{\docclsopt}[1]{\texttt{#1}}% document class option name

% Define a custom command for definitions and biconditional
\newcommand{\defn}[2]{\noindent\textbf{#1}:\ #2}
\let\biconditional\leftrightarrow

\begin{document}

\maketitle

\tableofcontents

\section{Course Description}
An introduction to digital system design, with an emphasis on
practical design techniques and circuit implementation.
\pagebreak

\section{Number Systems}
In daily life, we primarily interact with the familiar
base-10 numbers. However, when interaction with digital
systems, we must also concern ourselves with base-2, base-8,
base-16, and other bases which are friendly to binary
states. Unless completely unambiguous, the base of a number
is written as a right subscript such as $144_{10}$ for base-10
or $1001_{2}$ for base-2.

For binary numbers, each digit represents a power of two.
To convert a binary number to decimal, you sum the products of
each binary digit with its corresponding power of two. For
example, the binary number 1001 is calculated as
\begin{align}
    2^3 \times 1 + 2^2 \times 0 + 2^1 \times 0 + 2^0 \times 1 & = 8 + 0 + 0 + 1 \\
                                                              & = 9_{10}
\end{align}

To convert from hexadecimal to decimal, each digit represents a
power of sixteen. For instance, the hexadecimal number f1 is
calculated as
\begin{align}
    15 \times 161+1 \times 160 & = 240+1    \\
                               & = 241_{10}
\end{align}
where f represents the decimal value 15.
When converting to another base, reverse the process by dividing
the decimal number by the target base, recording the remainder,
and repeating with the quotient until it reaches zero.
The remainders give you the digits of the number in the new
base, read in reverse order.

To convert a decimal number into binary, for example, you
repeatedly divide the number by 2 and record the remainders.
For the decimal number 9, dividing by 2 gives a quotient of 4
and a remainder of 1. Dividing 4 by 2 gives a quotient of 2
and a remainder of 0. Dividing 2 by 2 gives a quotient of 1
and a remainder of 0, and finally, dividing 1 by 2 gives a
quotient of 0 and a remainder of 1. Reading the remainders
from bottom to top, the binary representation of 9 is 1001.

\end{document}

%\begin{center}
%    \begin{forest}
%        [0.INTMAX [1.16 [2.12 [3.10] [3.12]] [3.16 [4.14] [4.16]]][4.INTMAX [5.20 [6.18] [6.20]] [6.INTMAX]]]
%    \end{forest}    
%\end{center}