\section{Multiplexers}
A multiplexer (MUX) is a combinational logic circuit that
selects one of many input signals and forwards the selected
input to a single output.

A multiplexer has multiple input lines, a set of selection lines,
and a single output line. The selection lines determine which
input is forwarded to the output. For an $n$-to-1 multiplexer,
there are $n$ input lines and $\log_2 n$ selection lines.

\subsection*{2-to-1 Multiplexer}
The simplest multiplexer is a 2-to-1 multiplexer, which has two
inputs ($I_0, I_1$), one selection line ($S$), and one output ($Y$).
The function of a 2-to-1 MUX is defined as:
\[
    Y = S \cdot I_1 + \bar{S} \cdot I_0
\]
This means:
\begin{itemize}
    \item If $S = 0$, then $Y = I_0$.
    \item If $S = 1$, then $Y = I_1$.
\end{itemize}

\subsection*{4-to-1 Multiplexer}
A 4-to-1 multiplexer has four inputs ($I_0, I_1, I_2, I_3$), two
selection lines ($S_0, S_1$), and one output ($Y$). The output equation is:
\[
    Y = (\bar{S_1} \cdot \bar{S_0} \cdot I_0) + (\bar{S_1} \cdot S_0 \cdot I_1) + (S_1 \cdot \bar{S_0} \cdot I_2) + (S_1 \cdot S_0 \cdot I_3)
\]

\subsection*{Multiplexers in Digital Systems}
Multiplexers are used in:
\begin{itemize}
    \item Data selection and routing.
    \item Logic function implementation.
    \item Memory addressing.
    \item Communication systems.
\end{itemize}

\subsection*{Implementation Using Logic Gates}
A multiplexer can be implemented using basic logic gates such as
AND, OR, and NOT gates. For example, a 2-to-1 multiplexer can be
constructed using two AND gates, one OR gate, and one NOT gate.
