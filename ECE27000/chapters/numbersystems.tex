\section{Number Systems}
In daily life, we primarily interact with the familiar
base-10 numbers. However, when interaction with digital
systems, we must also concern ourselves with base-2, base-8,
base-16, and other bases which are friendly to binary
states. Unless completely unambiguous, the base of a number
is written as a right subscript such as $144_{10}$ for base-10
or $1001_{2}$ for base-2.

For binary numbers, each digit represents a power of two.
To convert a binary number to decimal, you sum the products of
each binary digit with its corresponding power of two. For
example, the binary number 1001 is calculated as
\begin{align}
    2^3 \times 1 + 2^2 \times 0 + 2^1 \times 0 + 2^0 \times 1 & = 8 + 0 + 0 + 1 \\
                                                              & = 9_{10}
\end{align}

To convert from hexadecimal to decimal, each digit represents a
power of sixteen. For instance, the hexadecimal number f1 is
calculated as
\begin{align}
    15 \times 161+1 \times 160 & = 240+1    \\
                               & = 241_{10}
\end{align}
where f represents the decimal value 15.
When converting to another base, reverse the process by dividing
the decimal number by the target base, recording the remainder,
and repeating with the quotient until it reaches zero.
The remainders give you the digits of the number in the new
base, read in reverse order.

To convert a decimal number into binary, for example, you
repeatedly divide the number by 2 and record the remainders.
For the decimal number 9, dividing by 2 gives a quotient of 4
and a remainder of 1. Dividing 4 by 2 gives a quotient of 2
and a remainder of 0. Dividing 2 by 2 gives a quotient of 1
and a remainder of 0, and finally, dividing 1 by 2 gives a
quotient of 0 and a remainder of 1. Reading the remainders
from bottom to top, the binary representation of 9 is 1001.

In SystemVerilog, numbers are written in format
\texttt{[size]'[base][number]}, for example:
\begin{itemize}
    \item 4'b1001 // binary, 9 in decimal, bit width 4 bits
    \item 8'hf1 //hex, equals 421, bit width 8 bits
    \item 3'o3 // octal, 3, bit width 3 bits
    \item 32'b1001\_1101\_0101\_1111 // binary, 40255, bit width 32 bits
\end{itemize}