\section{Power Consumption}
The power consumption of a CMOS circuit whose output is not changing is
called \emph{static power dissipation}. 
The power a CMOS circuit consumes during signal transitions is 
called \emph{dynamic power dissipation}.

In general dynamic power dissipation is much larger than static. 
One significant source of dissipation is output transitions. The 
power consumed as the voltage transitions is given by
\begin{equation}
    P_T = C_{PD} V_{CC}^2 f
\end{equation}
where 
\begin{itemize}
    \item $P_T$ is circuit's internal power dissipation due to output transitions
    \item $C_{PD}$ is power-dissipation capacitance, normally specified
    by the device manufacturer
    \item $V_{CC}$ is power supply voltage
    \item $f$ is transition frequency of the output signal
\end{itemize}
A second (and often more significant) source of CMOS power consumption 
is the capacitive load ($C_L$ ) on the output. 