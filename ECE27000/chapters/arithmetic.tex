\section{Arithmetic}

\subsection{Signed Numbers}
In SystemVerilog to this point, we have been working with \emph{unsigned}
(positive) numbers. We now introduce the concept of \emph{signed} binary
numbers, which will enable us to perform subtraction and other useful
operations.

Signed numbers are typically represented using the
\emph{two's complement} method. In this representation:
\begin{itemize}
    \item The most significant bit (MSB) is the \emph{sign bit},
          where 0 indicates a positive number and 1 indicates a negative
          number.
    \item The value of a signed number can be calculated as:
          \[
              \text{Value} = -2^{n-1} \cdot b_{n-1} + \sum_{i=0}^{n-2} 2^i \cdot b_i
          \]
          where $b_i$ represents the bits of the binary number.
\end{itemize}

For example, in a 4-bit two's complement system:
\begin{itemize}
    \item $0110$ represents $6$ (positive).
    \item $1010$ represents $-6$ (negative).
\end{itemize}

\subsection{Adders and Subtractors}
Adders and subtractors are fundamental components in digital
arithmetic. They are used to perform addition and subtraction
operations on binary numbers.

A \emph{half adder} is a combinational circuit that adds two
binary digits and produces a sum and a carry. The truth table
for a half adder is:

\begin{table}[h]
    \centering
    \begin{tabular}{|c|c|c|c|}
        \hline
        \textbf{A} & \textbf{B} & \textbf{Sum} & \textbf{Carry} \\
        \hline
        0          & 0          & 0            & 0              \\
        0          & 1          & 1            & 0              \\
        1          & 0          & 1            & 0              \\
        1          & 1          & 0            & 1              \\
        \hline
    \end{tabular}
    \caption{Truth table for a half adder}
\end{table}

The logic equations for the sum and carry are:
\[
    \text{Sum} = A \oplus B, \quad \text{Carry} = A \cdot B
\]

A \emph{full adder} extends the half adder by including a carry-in bit.
It adds three binary digits (A, B, and Carry-in) and produces a sum and
a carry-out. The logic equations for the full adder are:
\[
    \text{Sum} = A \oplus B \oplus \text{Carry-in}
\]
\[
    \text{Carry-out} = (A \cdot B) + (\text{Carry-in} \cdot (A \oplus B))
\]

Subtraction can be performed using an adder by taking the two's
complement of the number to be subtracted. For example, to compute
$A - B$, we calculate $A + (-B)$, where $-B$ is the two's complement
of $B$.

\subsection{Comparators}
Comparators are used to compare two binary numbers and determine
their relationship. The outputs of a comparator typically indicate
whether one number is less than, equal to, or greater than the other.

For two numbers $A$ and $B$:
\begin{itemize}
    \item $A < B$: Output is 1 if $A$ is less than $B$.
    \item $A = B$: Output is 1 if $A$ is equal to $B$.
    \item $A > B$: Output is 1 if $A$ is greater than $B$.
\end{itemize}

A simple 1-bit comparator can be implemented using logic gates:
\[
    A < B = \overline{A} \cdot B, \quad A = B = A \oplus B, \quad A > B = A \cdot \overline{B}
\]

\subsection{Shifters and Multipliers}
Shifters and multipliers are used for more complex arithmetic operations.

Shifters move the bits of a binary number to the left or right.
There are three main types of shifters:
\begin{itemize}
    \item \textbf{Logical Shift}: Shifts the bits left or right,
          filling the vacated positions with 0s.
    \item \textbf{Arithmetic Shift}: Preserves the sign bit when
          shifting right, filling the vacated positions with the sign bit.
    \item \textbf{Circular Shift (Rotate)}: Rotates the bits,
          wrapping around the bits that are shifted out.
\end{itemize}

For example, a logical left shift of $1010$ by 1 position results
in $0100$.

Multipliers perform binary multiplication. The simplest method is
repeated addition, but more efficient algorithms like the Booth
multiplier or Wallace tree are used in hardware implementations.

For example, multiplying $101$ (5 in decimal) by $11$ (3 in decimal)
yields $1111$ (15 in decimal).