\section{Decoders and Encoders}

\subsection{Decoders}
Decoders are combinational circuits that
convert binary information from $n$ input
lines to a maximum of $2^n$ unique output lines.
They are used in various applications such as
memory address decoding, data demultiplexing,
and seven-segment displays. A common
type of decoder is the 2-to-4 line decoder,
which has 2 input lines and 4 output lines.
The output lines are mutually exclusive,
meaning only one output line is active (high)
at any given time based on the binary value of
the inputs.

\subsection{Encoders}
Encoders are combinational circuits that perform
the inverse operation of decoders. They convert
$2^n$ input lines into an $n$-bit binary code.
Encoders are used in applications such as priority
encoding, data compression, and binary coding of
input signals. A common type of encoder is the 4-to-2
line encoder, which has 4 input lines and 2 output
lines. The encoder generates a binary code corresponding
to the active input line.

In practice, encoders often include additional
features such as priority encoding, where the
encoder outputs the binary code of the
highest-priority active input.