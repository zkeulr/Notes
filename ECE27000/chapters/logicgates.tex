\section{Logic Gates}
We represent boolean operations in circuit diagrams
with gates, whose symbols and truth tables are below.
\begin{itemize}
    \item Buffer \\
          \begin{tikzpicture}[circuit logic US]
              \node (buffer) [buffer gate, draw, logic gate inputs=1, anchor=output] {};
          \end{tikzpicture}
          \\
          \begin{tabular}{|c|c|}
              \hline
              A & Output \\
              \hline
              0 & 0      \\
              1 & 1      \\
              \hline
          \end{tabular}
    \item AND \\
          \begin{tikzpicture}[circuit logic US][circuit logic US]
              \node (and) [and gate, draw, logic gate inputs=nn] {};
          \end{tikzpicture}
          \\
          \begin{tabular}{|c|c|c|}
              \hline
              A & B & Output \\
              \hline
              0 & 0 & 0      \\
              0 & 1 & 0      \\
              1 & 0 & 0      \\
              1 & 1 & 1      \\
              \hline
          \end{tabular}
    \item OR \\
          \begin{tikzpicture}[circuit logic US]
              \node (or) [or gate, draw, logic gate inputs=nn] {};
          \end{tikzpicture}
          \\
          \begin{tabular}{|c|c|c|}
              \hline
              A & B & Output \\
              \hline
              0 & 0 & 0      \\
              0 & 1 & 1      \\
              1 & 0 & 1      \\
              1 & 1 & 1      \\
              \hline
          \end{tabular}
    \item NOT \\
          \begin{tikzpicture}[circuit logic US]
              \node (not) [not gate, draw, logic gate inputs=n] {};
          \end{tikzpicture}
          \\
          \begin{tabular}{|c|c|}
              \hline
              A & Output \\
              \hline
              0 & 1      \\
              1 & 0      \\
              \hline
          \end{tabular}
    \item NAND \\
          \begin{tikzpicture}[circuit logic US]
              \node (nand) [nand gate, draw, logic gate inputs=nn] {};
          \end{tikzpicture}
          \\
          \begin{tabular}{|c|c|c|}
              \hline
              A & B & Output \\
              \hline
              0 & 0 & 1      \\
              0 & 1 & 1      \\
              1 & 0 & 1      \\
              1 & 1 & 0      \\
              \hline
          \end{tabular}
    \item NOR \\
          \begin{tikzpicture}[circuit logic US]
              \node (nor) [nor gate, draw, logic gate inputs=nn] {};
          \end{tikzpicture}
          \\
          \begin{tabular}{|c|c|c|}
              \hline
              A & B & Output \\
              \hline
              0 & 0 & 1      \\
              0 & 1 & 0      \\
              1 & 0 & 0      \\
              1 & 1 & 0      \\
              \hline
          \end{tabular}
    \item XOR \\
          \begin{tikzpicture}[circuit logic US]
              \node (xor) [xor gate, draw, logic gate inputs=nn] {};
          \end{tikzpicture}
          \\
          \begin{tabular}{|c|c|c|}
              \hline
              A & B & Output \\
              \hline
              0 & 0 & 0      \\
              0 & 1 & 1      \\
              1 & 0 & 1      \\
              1 & 1 & 0      \\
              \hline
          \end{tabular}
    \item XNOR \\
          \begin{tikzpicture}[circuit logic US]
              \node (xnor) [xnor gate, draw, logic gate inputs=nn] {};
          \end{tikzpicture}
          \\
          \begin{tabular}{|c|c|c|}
              \hline
              A & B & Output \\
              \hline
              0 & 0 & 1      \\
              0 & 1 & 0      \\
              1 & 0 & 0      \\
              1 & 1 & 1      \\
              \hline
          \end{tabular}
\end{itemize}