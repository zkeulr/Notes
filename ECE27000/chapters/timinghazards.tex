\section{Timing Hazards}

Timing hazards occur in digital circuits when changes in input
signals cause changes in the output to unexpected values before
settling to the expected value. They occur because of propagation
delays.

\subsection{Static Hazards}
Static hazards occur when an output temporarily glitches to an
incorrect value before settling. They are classified as:
\begin{itemize}
    \item \textbf{Static-1 Hazard}: The output should remain at logic `1`, but briefly drops to `0`.
    \item \textbf{Static-0 Hazard}: The output should remain at logic `0`, but briefly rises to `1`.
\end{itemize}
These hazards arise due to differing propagation delays through alternative logic paths.

The circuit in figure \ref{fig:hazard_example} may exhibit a hazard when
inputs change due to different propagation delays.
\begin{figure}[h]
    \centering
    \begin{circuitikz}
        \node[and port, number inputs=1] (A) at (0, 0) {};
        \node at (A) [ocirc,fill=black] {};
        \node[left] at (A.in 1) {\(x_1\)};
        \node at (A.left) [yshift=8pt,ocirc,fill=black] {};
        \node[left] at (A.in 2) {\(x_2\)};

        \node[or port] (B) at (3,-.65) {};
        \node[left] at (B.in 2-| A.in 1) (x3) {\(x_3\)};
        \node[right] at (B.out) {\((x_1'x_2)'+x_3\)};

        \draw (A.out)  --++ (1,0) |-  (B.in 1);
        \draw (x3)  --  (B.in 2);
    \end{circuitikz}
    \caption{Hazardous Circuit}
    \label{fig:hazard_example}
\end{figure}

\subsection{Dynamic Hazards}
Dynamic hazards occur when an output makes multiple transitions
before settling to the correct value. This happens when there are
multiple paths with differing delays, leading to oscillations in the
output before stabilization.

\subsection{Mitigation Techniques}

To minimize timing hazards, designers employ several techniques:
\begin{itemize}
    \item \textbf{Redundant Logic}: Adding extra logic gates to ensure that different paths remain synchronized.
    \item \textbf{Synchronizing Inputs}: Using flip-flops and registers to control when signals change.
    \item \textbf{Glitch Filtering}: Employing filters or careful circuit layout to suppress transient glitches.
    \item \textbf{Delay Balancing}: Ensuring all paths experience similar propagation delays to prevent unintended transitions.
\end{itemize}