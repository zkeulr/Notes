\section{K-Maps}

A Karnaugh Map (K-Map) is a visual representation used to simplify Boolean expressions and minimize logic functions. It is a method to perform Boolean algebra simplifications by organizing truth table values into a grid format, allowing easy identification of common terms.
K-Maps are structured as grids where each cell corresponds to a specific combination of input variables. The number of cells in a K-Map depends on the number of variables:

\begin{itemize}
    \item 2-variable K-Map: $2^2 = 4$ cells
    \item 3-variable K-Map: $2^3 = 8$ cells
    \item 4-variable K-Map: $2^4 = 16$ cells
\end{itemize}

For a function with two variables ($A$ and $B$), the K-Map is a $2 \times 2$ grid:

\begin{center}
    \begin{tabular}{|c|c|c|}
        \hline
        AB & 0      & 1      \\
        \hline
        0  & F(0,0) & F(0,1) \\
        \hline
        1  & F(1,0) & F(1,1) \\
        \hline
    \end{tabular}
\end{center}

For three variables ($A$, $B$, $C$), the K-Map has $2^3 = 8$ cells:

\begin{center}
    \begin{tabular}{|c|c|c|c|}
        \hline
        AB \textbackslash C & 0        & 1        \\
        \hline
        00                  & F(0,0,0) & F(0,0,1) \\
        \hline
        01                  & F(0,1,0) & F(0,1,1) \\
        \hline
        11                  & F(1,1,0) & F(1,1,1) \\
        \hline
        10                  & F(1,0,0) & F(1,0,1) \\
        \hline
    \end{tabular}
\end{center}

The goal of a K-Map is to group adjacent cells containing
1s into power-of-two groups (1, 2, 4, etc.), forming simplified expressions.
Groups can wrap around the edges of the K-Map.

An example would be illustrative. Consider the K-Map
in figure \ref{fig:kmapexample}.
\begin{center}
    \begin{figure}
        \begin{center}
            \begin{tabular}{|c|c|c|c|c|c|}
                \hline
                A \textbackslash BC & 00 & 01 & 11 & 10 \\
                \hline
                0                   & 0  & 1  & 1  & 0  \\
                \hline
                1                   & 0  & 0  & 1  & 1  \\
                \hline
            \end{tabular}
        \end{center}
        \caption{K-Map Example}
        \label{fig:kmapexample}
    \end{figure}
\end{center}
There are two groups, the two 1s in the first row
and the two 1s in the second. From the top group
we get the term $A'B'C + A'BC = A'C$. From the bottom group we
get the term $ABC + ABC' = AB$, and summing these product terms
we get the final expression $A'C + AB$.
