\section{Boolean Algebra}
Computers operate in binary. To represent the state of a
computer we require a suitable mathematical framework,
provided by boolean algebra. In boolean algebra, variables 
can only take on two values: 0 and 1. 

\begin{table}[h]
    \centering
    \begin{tabular}{|l|l|}
        \hline
        \textbf{Rule}           & \textbf{Expression}                                                                            \\ \hline
        Commutativity           & $X + Y = Y + X$                                                                                \\
                                & $X \cdot Y = Y \cdot X$                                                                        \\ \hline
        Associativity           & $(X + Y) + Z = X + (Y + Z)$                                                                    \\
                                & $(X \cdot Y) \cdot Z = X \cdot (Y \cdot Z)$                                                    \\ \hline
        Distributivity          & $X \cdot Y + X \cdot Z = X \cdot (Y + Z)$                                                      \\
                                & $(X + Y) \cdot (X + Z) = X + Y \cdot Z$                                                        \\ \hline
        Covering                & $X + X \cdot Y = X$                                                                            \\
                                & $X \cdot (X + Y) = X$                                                                          \\ \hline
        Combining               & $X \cdot Y + X \cdot Y = X$                                                                    \\
                                & $(X + Y) \cdot (X + Y) = X$                                                                    \\ \hline
        Consensus               & $X \cdot Y + X \cdot Z + Y \cdot Z = X \cdot Y + X' \cdot Z$                                   \\
                                & $(X + Y) \cdot (X + Z) \cdot (Y + Z) = (X + Y) \cdot (X + Z)$                                  \\ \hline
        Generalized Idempotency & $X + X + \dots + X = X$                                                                        \\
                                & $X \cdot X \cdot \dots \cdot X = X$                                                            \\ \hline
        DeMorgan's Theorems     & $(X_1 \cdot X_2 \cdot \dots \cdot X_n)' = X_1' + X_2' + \dots + X_n'$                          \\
                                & $(X_1 + X_2 + \dots + X_n)' = X_1' \cdot X_2' \cdot \dots \cdot X_n'$                          \\ \hline
        Generalized DeMorgan's  & $F(X_1, X_2, \dots, X_n, +, \cdot) = F(X_1, X_2, \dots, X_n, \cdot, +)'$                       \\ \hline
        Shannon's Expansion     & $F(X_1, X_2, \dots, X_n) = X_1 \cdot F(1, X_2, \dots, X_n) + X_1' \cdot F(0, X_2, \dots, X_n)$ \\
                                & $F(X_1, X_2, \dots, X_n) = [X_1 + F(0, X_2, \dots, X_n)] \cdot [X_1' + F(1, X_2, \dots, X_n)]$ \\ \hline
    \end{tabular}
    \caption{Boolean Algebra}
    \label{tab:boolean_rules}
\end{table}

An interesting and useful property in boolean algebra is "duality", where 
replacing all ANDs with ORs and all 1s with 0s gives a valid and 
equivalent theorem. For instance,
\begin{table}[h]
    \centering
    \begin{tabular}{|c|c|}
        \hline
        X AND 0 = 0 & X OR 1 = 1 \\ \hline
        X OR 0 = X & X AND 1 = X \\ 
        \hline                                                                       \\
    \end{tabular}
    \caption{Boolean Duality}
    \label{tab:boolean_duality}
\end{table}

Any logic can be implemented using just the following:
\begin{itemize}
    \item AND, OR, and NOT gates
    \item NAND gates
    \item NOR gates
\end{itemize}