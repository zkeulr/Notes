\section{CMOS}
Complementary Metal-Oxide-Semiconductor (CMOS) 
technology is the dominant semiconductor technology 
for modern integrated circuits. CMOS combines both n-type 
(NMOS) and p-type (PMOS) metal-oxide-semiconductor field-effect
transistors (MOSFETs) to create the familiar logic gates such as 
AND, NOT, XOR, etc. 

\subsection{PMOS}
\begin{figure}[h]
    \centering
    \begin{circuitikz}
        \draw (0,0) node[pmos](P){};
        \draw (P.gate) to[short] ++(-0.5,0) node[left]{Gate};
        \draw (P.source) to[short] ++(0,0.5) node[above]{Source ($V_{DD}$)};
        \draw (P.drain) to[short] ++(0,-0.5) node[below]{Drain};
    \end{circuitikz}
    \caption{PMOS transistor circuit symbol}
    \label{fig:pmos}
\end{figure}

PMOS transistors consist of:
\begin{itemize}
    \item p+ source and drain regions
    \item n-type substrate (body)
    \item SiO\textsubscript{2} gate dielectric
    \item Polysilicon gate electrode
\end{itemize}

PMOS operates with negative gate-to-source voltage ($V_{GS}$):
\begin{itemize}
    \item \textbf{Cut-off Region} ($V_{GS} > V_{th,p}$):
    \begin{equation*}
        I_D = 0
    \end{equation*}
    \item \textbf{Linear Region} ($V_{GS} \leq V_{th,p}$ and $V_{DS} \geq V_{GS} - V_{th,p}$):
    \begin{equation*}
        I_D = -\mu_p C_{ox} \frac{W}{L} \left[(V_{GS} - V_{th,p})V_{DS} - \frac{V_{DS}^2}{2}\right]
    \end{equation*}
    \item \textbf{Saturation Region} ($V_{GS} \leq V_{th,p}$ and $V_{DS} < V_{GS} - V_{th,p}$):
    \begin{equation*}
        I_D = -\frac{1}{2} \mu_p C_{ox} \frac{W}{L} (V_{GS} - V_{th,p})^2
    \end{equation*}
\end{itemize}

\subsection{NMOS}
\begin{figure}[h]
    \centering
    \begin{circuitikz}
        \draw (0,0) node[nmos](N){};
        \draw (N.gate) to[short] ++(-0.5,0) node[left]{Gate};
        \draw (N.source) to[short] ++(0,-0.5) node[below]{Source (GND)};
        \draw (N.drain) to[short] ++(0,0.5) node[above]{Drain};
    \end{circuitikz}
    \caption{NMOS transistor circuit symbol}
    \label{fig:nmos}
\end{figure}

NMOS transistors consist of:
\begin{itemize}
    \item n+ source and drain regions
    \item p-type substrate (body)
    \item SiO\textsubscript{2} gate dielectric
    \item Polysilicon gate electrode
\end{itemize}

NMOS operates with positive gate-to-source voltage ($V_{GS}$):
\begin{itemize}
    \item \textbf{Cut-off Region} ($V_{GS} < V_{th,n}$):
    \begin{equation*}
        I_D = 0
    \end{equation*}
    \item \textbf{Linear Region} ($V_{GS} \geq V_{th,n}$ and $V_{DS} \leq V_{GS} - V_{th,n}$):
    \begin{equation*}
        I_D = \mu_n C_{ox} \frac{W}{L} \left[(V_{GS} - V_{th,n})V_{DS} - \frac{V_{DS}^2}{2}\right]
    \end{equation*}
    \item \textbf{Saturation Region} ($V_{GS} \geq V_{th,n}$ and $V_{DS} > V_{GS} - V_{th,n}$):
    \begin{equation*}
        I_D = \frac{1}{2} \mu_n C_{ox} \frac{W}{L} (V_{GS} - V_{th,n})^2
    \end{equation*}
\end{itemize}
\begin{table}[h]
    \centering
    \begin{tabular}{|l|l|l|}
        \hline
        \textbf{Parameter} & \textbf{PMOS} & \textbf{NMOS} \\
        \hline
        Majority Carrier & Holes & Electrons \\
        Substrate Type & n-type & p-type \\
        Threshold Voltage & Negative & Positive \\
        Mobility ($\mu$) & Lower ($\approx 150 \frac{cm^2}{V_s}$) & Higher ($\approx 400 \frac{cm^2}{V_s}$) \\
        Speed & Slower & Faster \\
        \hline
    \end{tabular}
    \caption{PMOS vs NMOS characteristics}
    \label{tab:comparison}
\end{table}